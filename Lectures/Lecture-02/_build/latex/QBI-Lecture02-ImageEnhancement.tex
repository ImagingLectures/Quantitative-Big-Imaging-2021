%% Generated by Sphinx.
\def\sphinxdocclass{report}
\documentclass[letterpaper,10pt,english]{sphinxmanual}
\ifdefined\pdfpxdimen
   \let\sphinxpxdimen\pdfpxdimen\else\newdimen\sphinxpxdimen
\fi \sphinxpxdimen=.75bp\relax

\PassOptionsToPackage{warn}{textcomp}
\usepackage[utf8]{inputenc}
\ifdefined\DeclareUnicodeCharacter
% support both utf8 and utf8x syntaxes
  \ifdefined\DeclareUnicodeCharacterAsOptional
    \def\sphinxDUC#1{\DeclareUnicodeCharacter{"#1}}
  \else
    \let\sphinxDUC\DeclareUnicodeCharacter
  \fi
  \sphinxDUC{00A0}{\nobreakspace}
  \sphinxDUC{2500}{\sphinxunichar{2500}}
  \sphinxDUC{2502}{\sphinxunichar{2502}}
  \sphinxDUC{2514}{\sphinxunichar{2514}}
  \sphinxDUC{251C}{\sphinxunichar{251C}}
  \sphinxDUC{2572}{\textbackslash}
\fi
\usepackage{cmap}
\usepackage[T1]{fontenc}
\usepackage{amsmath,amssymb,amstext}
\usepackage{babel}



\usepackage{times}
\expandafter\ifx\csname T@LGR\endcsname\relax
\else
% LGR was declared as font encoding
  \substitutefont{LGR}{\rmdefault}{cmr}
  \substitutefont{LGR}{\sfdefault}{cmss}
  \substitutefont{LGR}{\ttdefault}{cmtt}
\fi
\expandafter\ifx\csname T@X2\endcsname\relax
  \expandafter\ifx\csname T@T2A\endcsname\relax
  \else
  % T2A was declared as font encoding
    \substitutefont{T2A}{\rmdefault}{cmr}
    \substitutefont{T2A}{\sfdefault}{cmss}
    \substitutefont{T2A}{\ttdefault}{cmtt}
  \fi
\else
% X2 was declared as font encoding
  \substitutefont{X2}{\rmdefault}{cmr}
  \substitutefont{X2}{\sfdefault}{cmss}
  \substitutefont{X2}{\ttdefault}{cmtt}
\fi


\usepackage[Bjarne]{fncychap}
\usepackage[,numfigreset=1,mathnumfig]{sphinx}

\fvset{fontsize=\small}
\usepackage{geometry}


% Include hyperref last.
\usepackage{hyperref}
% Fix anchor placement for figures with captions.
\usepackage{hypcap}% it must be loaded after hyperref.
% Set up styles of URL: it should be placed after hyperref.
\urlstyle{same}


\usepackage{sphinxmessages}




\title{Quantitative Big Imaging - Image enhancement}
\date{Mar 14, 2021}
\release{}
\author{Anders Kaestner}
\newcommand{\sphinxlogo}{\vbox{}}
\renewcommand{\releasename}{}
\makeindex
\begin{document}

\pagestyle{empty}
\sphinxmaketitle
\pagestyle{plain}
\sphinxtableofcontents
\pagestyle{normal}
\phantomsection\label{\detokenize{02-ImageEnhancement::doc}}


\sphinxAtStartPar
\sphinxstylestrong{Quantitative Big Imaging} ETHZ: 227\sphinxhyphen{}0966\sphinxhyphen{}00L




\chapter{Todays lecture}
\label{\detokenize{02-ImageEnhancement:todays-lecture}}\begin{itemize}
\item {} 
\sphinxAtStartPar
Imperfect images and noise

\item {} 
\sphinxAtStartPar
Filters

\item {} 
\sphinxAtStartPar
Advanced filters

\item {} 
\sphinxAtStartPar
Evaluation workflows

\end{itemize}


\section{We need some modules}
\label{\detokenize{02-ImageEnhancement:we-need-some-modules}}
\sphinxAtStartPar
These modules are needed to run the python cells in this lecture.

\begin{sphinxVerbatim}[commandchars=\\\{\}]
\PYG{o}{\PYGZpc{}}\PYG{k}{matplotlib} inline
\PYG{k+kn}{import} \PYG{n+nn}{matplotlib}\PYG{n+nn}{.}\PYG{n+nn}{pyplot} \PYG{k}{as} \PYG{n+nn}{plt}
\PYG{k+kn}{import} \PYG{n+nn}{numpy} \PYG{k}{as} \PYG{n+nn}{np}
\PYG{k+kn}{import} \PYG{n+nn}{skimage} \PYG{k}{as} \PYG{n+nn}{ski}
\PYG{k+kn}{import} \PYG{n+nn}{skimage}\PYG{n+nn}{.}\PYG{n+nn}{io} \PYG{k}{as} \PYG{n+nn}{io}
\PYG{k+kn}{from} \PYG{n+nn}{skimage}\PYG{n+nn}{.}\PYG{n+nn}{morphology} \PYG{k+kn}{import} \PYG{n}{disk}
\PYG{k+kn}{import} \PYG{n+nn}{scipy}\PYG{n+nn}{.}\PYG{n+nn}{ndimage} \PYG{k}{as} \PYG{n+nn}{ndimage}
\PYG{k+kn}{import} \PYG{n+nn}{matplotlib}        \PYG{k}{as} \PYG{n+nn}{mpl}
\PYG{n}{mpl}\PYG{o}{.}\PYG{n}{rcParams}\PYG{p}{[}\PYG{l+s+s1}{\PYGZsq{}}\PYG{l+s+s1}{figure.dpi}\PYG{l+s+s1}{\PYGZsq{}}\PYG{p}{]} \PYG{o}{=} \PYG{l+m+mi}{100}
\end{sphinxVerbatim}

\begin{sphinxVerbatim}[commandchars=\\\{\}]
\PYG{g+gt}{\PYGZhy{}\PYGZhy{}\PYGZhy{}\PYGZhy{}\PYGZhy{}\PYGZhy{}\PYGZhy{}\PYGZhy{}\PYGZhy{}\PYGZhy{}\PYGZhy{}\PYGZhy{}\PYGZhy{}\PYGZhy{}\PYGZhy{}\PYGZhy{}\PYGZhy{}\PYGZhy{}\PYGZhy{}\PYGZhy{}\PYGZhy{}\PYGZhy{}\PYGZhy{}\PYGZhy{}\PYGZhy{}\PYGZhy{}\PYGZhy{}\PYGZhy{}\PYGZhy{}\PYGZhy{}\PYGZhy{}\PYGZhy{}\PYGZhy{}\PYGZhy{}\PYGZhy{}\PYGZhy{}\PYGZhy{}\PYGZhy{}\PYGZhy{}\PYGZhy{}\PYGZhy{}\PYGZhy{}\PYGZhy{}\PYGZhy{}\PYGZhy{}\PYGZhy{}\PYGZhy{}\PYGZhy{}\PYGZhy{}\PYGZhy{}\PYGZhy{}\PYGZhy{}\PYGZhy{}\PYGZhy{}\PYGZhy{}\PYGZhy{}\PYGZhy{}\PYGZhy{}\PYGZhy{}\PYGZhy{}\PYGZhy{}\PYGZhy{}\PYGZhy{}\PYGZhy{}\PYGZhy{}\PYGZhy{}\PYGZhy{}\PYGZhy{}\PYGZhy{}\PYGZhy{}\PYGZhy{}\PYGZhy{}\PYGZhy{}\PYGZhy{}\PYGZhy{}}
\PYG{n+ne}{ModuleNotFoundError}\PYG{g+gWhitespace}{                       }Traceback (most recent call last)
\PYG{o}{\PYGZlt{}}\PYG{n}{ipython}\PYG{o}{\PYGZhy{}}\PYG{n+nb}{input}\PYG{o}{\PYGZhy{}}\PYG{l+m+mi}{1}\PYG{o}{\PYGZhy{}}\PYG{n}{ffbe9569d72e}\PYG{o}{\PYGZgt{}} \PYG{o+ow}{in} \PYG{o}{\PYGZlt{}}\PYG{n}{module}\PYG{o}{\PYGZgt{}}
\PYG{g+gWhitespace}{      }\PYG{l+m+mi}{2} \PYG{k+kn}{import} \PYG{n+nn}{matplotlib}\PYG{n+nn}{.}\PYG{n+nn}{pyplot} \PYG{k}{as} \PYG{n+nn}{plt}
\PYG{g+gWhitespace}{      }\PYG{l+m+mi}{3} \PYG{k+kn}{import} \PYG{n+nn}{numpy} \PYG{k}{as} \PYG{n+nn}{np}
\PYG{n+ne}{\PYGZhy{}\PYGZhy{}\PYGZhy{}\PYGZhy{}\PYGZgt{} }\PYG{l+m+mi}{4} \PYG{k+kn}{import} \PYG{n+nn}{skimage} \PYG{k}{as} \PYG{n+nn}{ski}
\PYG{g+gWhitespace}{      }\PYG{l+m+mi}{5} \PYG{k+kn}{import} \PYG{n+nn}{skimage}\PYG{n+nn}{.}\PYG{n+nn}{io} \PYG{k}{as} \PYG{n+nn}{io}
\PYG{g+gWhitespace}{      }\PYG{l+m+mi}{6} \PYG{k+kn}{from} \PYG{n+nn}{skimage}\PYG{n+nn}{.}\PYG{n+nn}{morphology} \PYG{k+kn}{import} \PYG{n}{disk}

\PYG{n+ne}{ModuleNotFoundError}: No module named \PYGZsq{}skimage\PYGZsq{}
\end{sphinxVerbatim}


\chapter{Measurements are rarely perfect}
\label{\detokenize{02-ImageEnhancement:measurements-are-rarely-perfect}}


\sphinxAtStartPar
There is no perfect measurement. This is also true for neutron imaging. The ideal image is the sample is distorted for many reasons. The figure below shows how an image of a sample can look after passing though the acquisition system. These quailty degradations will have an impact on the analysis of the image data. In some cases, it is possible to correct for some of these artifacts using classing image processing techniques. There are however also cases that require extra effort to correct the artifacts.

\begin{figure}[htbp]
\centering
\capstart

\noindent\sphinxincludegraphics[scale=0.75]{{imperfect_imaging_system}.pdf}
\caption{Schematic showing the error contributions in an imperfect imaging system.}\label{\detokenize{02-ImageEnhancement:id4}}\end{figure}


\section{Factors affecting the image quality}
\label{\detokenize{02-ImageEnhancement:factors-affecting-the-image-quality}}
\sphinxAtStartPar
No measurement system is perfect and therefore, you will also not get perfect images of the sample you measured. In the figure you can see the slice as it would be in a perfect world to the left and what you actually would measure with an imaging system. Some are of higher quality than others but there are still a collection of factors that have an impact on the image quality.

\sphinxAtStartPar
The list below provides some factors that affect the quality of the acquired images. Most of them can be handled by changing the imaging configuration in some sense. It can however be that the sample or process observed put limitiations on how much the acquisition can be tuned to obtain the perfect image.
\begin{itemize}
\item {} 
\sphinxAtStartPar
Resolution (Imaging system transfer functions)

\item {} 
\sphinxAtStartPar
Noise

\item {} 
\sphinxAtStartPar
Contrast

\item {} 
\sphinxAtStartPar
Inhomogeneous contrast

\item {} 
\sphinxAtStartPar
Artifacts

\end{itemize}


\subsection{Resolution}
\label{\detokenize{02-ImageEnhancement:resolution}}
\sphinxAtStartPar
The resolution is primarily determined optical transfer function of the imaging system. The actual resolution is given by the extents of the sample and how much the detector needs to capture in one image. This gives the field of view and given the number pixels in the used detector it is possible to calculate the pixel size. The pixel size limits the size of the smallest feature in the image that can be detected. The scintillator, which is used to convert neutrons into visible light, is chosen to
\begin{enumerate}
\sphinxsetlistlabels{\arabic}{enumi}{enumii}{}{.}%
\item {} 
\sphinxAtStartPar
match the sampling rate given by the pixel size.

\item {} 
\sphinxAtStartPar
provide sufficient neutron capture to obtain sufficient light output for a given exposure time.

\end{enumerate}


\subsection{Noise}
\label{\detokenize{02-ImageEnhancement:noise}}
\sphinxAtStartPar
An imaging system has many noise sources, each with its own distribution e.g.
\begin{enumerate}
\sphinxsetlistlabels{\arabic}{enumi}{enumii}{}{.}%
\item {} 
\sphinxAtStartPar
Neutron statistics \sphinxhyphen{} how many neutrons are collected in a pixel. This noise is Poisson distributed.

\item {} 
\sphinxAtStartPar
Photon statistics \sphinxhyphen{} how many photons are produced by each neutron. This noise is also Poisson distributed.

\item {} 
\sphinxAtStartPar
Thermal noise from the electronics which has a Gaussian distribution.

\item {} 
\sphinxAtStartPar
Digitation noise from converting the charges collected for the photons into digital numbers that can be transfered and stored by a computer, this noise has a binominal distribution.

\end{enumerate}

\sphinxAtStartPar
The neutron statistics are mostly dominant in neutron imaging but in some cases it could also be that the photon statistics play a role.


\subsection{Contrast}
\label{\detokenize{02-ImageEnhancement:contrast}}
\sphinxAtStartPar
The contrast in the sample is a consequence of
\begin{enumerate}
\sphinxsetlistlabels{\arabic}{enumi}{enumii}{}{.}%
\item {} 
\sphinxAtStartPar
how well the sample transmits the chosen radiation type. For neutrons you obtain good contrast from materials containing hydrogen or lithium while many metals are more transparent.

\item {} 
\sphinxAtStartPar
the amount of a specific element or material represented in a unit cell, e.g. a pixel (radiograph) or a voxel (tomography).

\end{enumerate}

\sphinxAtStartPar
The objective of many experiments is to quantify the amount of a specific material. This could for example be the amount of water in a porous medium.

\sphinxAtStartPar
Good contrast between different image features is important if you want to segment them to make conclusions about the image content. Therefore, the radiation type should be chosen to provide the best contrast between the features.


\subsection{Inhomogeneous contrast}
\label{\detokenize{02-ImageEnhancement:inhomogeneous-contrast}}
\sphinxAtStartPar
The contrast in the raw radiograph depends much on the beam profile. These variations are easily corrected by normalizing the images by an open beam or flat field image.
\begin{itemize}
\item {} 
\sphinxAtStartPar
\sphinxstylestrong{Biases introduced by scattering} Scattering is the dominant interaction for many materials use in neutron imaging. This means that neutrons that are not passing straight though the sample are scattered and contribute to a background cloud of neutrons that build up a bias of neutron that are also detected and contribute to the

\item {} 
\sphinxAtStartPar
\sphinxstylestrong{Biases from beam hardening} is a problem that is more present in x\sphinxhyphen{}ray imaging and is caused by that fact that the attenuation coefficient depends on the energy of the radiation. Higher energies have lower attenuation coefficient, thus will high energies penetrate the thicker samples than lower energies. This can be seen when a polychromatic beam is used.

\end{itemize}


\subsubsection{Artifacts}
\label{\detokenize{02-ImageEnhancement:artifacts}}
\sphinxAtStartPar
Many images suffer from outliers caused by stray rays hitting the detector. Typical artefacts in tomography data are
\begin{itemize}
\item {} 
\sphinxAtStartPar
Lines, which are caused by outlier spots that only appear in single projections. These spot appear as lines in the reconstructed images.

\item {} 
\sphinxAtStartPar
Rings are caused by stuck pixels which have the same value in a projections.

\end{itemize}


\section{A typical processing chain}
\label{\detokenize{02-ImageEnhancement:a-typical-processing-chain}}
\sphinxAtStartPar
Traditionally, the processing of image data can be divided into a series of sub tasks that provide the final result.

\begin{figure}[htbp]
\centering
\capstart

\noindent\sphinxincludegraphics[scale=1.0]{{image_proc_chain}.pdf}
\caption{Typical steps of an image processing work flow.}\label{\detokenize{02-ImageEnhancement:id5}}\end{figure}
\begin{itemize}
\item {} 
\sphinxAtStartPar
\sphinxstylestrong{Acquisition} The data must obviously be acquired and stored. There are cases when simulated data is used. Then, the acquisition is replaced by the process to simulate the data.

\item {} 
\sphinxAtStartPar
\sphinxstylestrong{Enhancement} The raw data is usually not ready to be processed in the form is comes from the acquisition. It usually has noise and artifacts as we saw on the previous slide. The enhancement step suppresses unwanted information in the data.

\item {} 
\sphinxAtStartPar
\sphinxstylestrong{Segmentation} The segmenation identifies different regions based on different features such as intensity distribution and shape.

\item {} 
\sphinxAtStartPar
\sphinxstylestrong{Post processing} After segmentation, there may be falsely identified regions. These are removed in a post processing step.

\item {} 
\sphinxAtStartPar
\sphinxstylestrong{Evaluation} The last step of the process is to make conclusions based on the image data. It could be modelling material distrbutions, measuring shapes etc.

\end{itemize}



\sphinxAtStartPar
Today’s lecture will focus on \sphinxstylestrong{enhancement}


\chapter{Noise and artifacts}
\label{\detokenize{02-ImageEnhancement:noise-and-artifacts}}
\sphinxAtStartPar
Noise is in very general terms for the unwanted information in a signal. More specifically, we are talking about random contributions that obscure the image information we are interested in.


\section{Noise types}
\label{\detokenize{02-ImageEnhancement:noise-types}}
\sphinxAtStartPar
Noise can have many different characteristics. In general, it is driven by a random distribution.
\begin{itemize}
\item {} 
\sphinxAtStartPar
Spatially uncorrelated noise

\end{itemize}

\sphinxAtStartPar
With spatially uncorrelated noise each pixel has a random value which is independend of the pixel neighborhood. This is also the easiest noise type to simulate.
\begin{itemize}
\item {} 
\sphinxAtStartPar
Event noise

\end{itemize}

\sphinxAtStartPar
The even noise has a random activation function that triggers the event of each pixel with some probabilty. This noise type produces spots randomly distributed over the image. The spots may also have a randomw intensity.
\begin{itemize}
\item {} 
\sphinxAtStartPar
Stuctured noise

\end{itemize}

\sphinxAtStartPar
The structured noise depends on the values of the pixel neighborhood and is thus spatially correlated. It is mostly driven by an uncorrelated noise source which is blurred by a weighted combination of the neighborhood.

\sphinxAtStartPar
The figure below shows examples of the three noise types.

\begin{sphinxVerbatim}[commandchars=\\\{\}]
\PYG{n}{plt}\PYG{o}{.}\PYG{n}{figure}\PYG{p}{(}\PYG{n}{figsize}\PYG{o}{=}\PYG{p}{[}\PYG{l+m+mi}{15}\PYG{p}{,}\PYG{l+m+mi}{6}\PYG{p}{]}\PYG{p}{)}\PYG{p}{;}
\PYG{n}{plt}\PYG{o}{.}\PYG{n}{suptitle}\PYG{p}{(}\PYG{l+s+s1}{\PYGZsq{}}\PYG{l+s+s1}{Noise examples}\PYG{l+s+s1}{\PYGZsq{}}\PYG{p}{)}
\PYG{n}{plt}\PYG{o}{.}\PYG{n}{subplot}\PYG{p}{(}\PYG{l+m+mi}{1}\PYG{p}{,}\PYG{l+m+mi}{3}\PYG{p}{,}\PYG{l+m+mi}{1}\PYG{p}{)}\PYG{p}{;}\PYG{n}{plt}\PYG{o}{.}\PYG{n}{imshow}\PYG{p}{(}\PYG{n}{np}\PYG{o}{.}\PYG{n}{random}\PYG{o}{.}\PYG{n}{normal}\PYG{p}{(}\PYG{l+m+mi}{0}\PYG{p}{,}\PYG{l+m+mi}{1}\PYG{p}{,}\PYG{p}{[}\PYG{l+m+mi}{100}\PYG{p}{,}\PYG{l+m+mi}{100}\PYG{p}{]}\PYG{p}{)}\PYG{p}{)}\PYG{p}{;} \PYG{n}{plt}\PYG{o}{.}\PYG{n}{title}\PYG{p}{(}\PYG{l+s+s1}{\PYGZsq{}}\PYG{l+s+s1}{Gaussian}\PYG{l+s+s1}{\PYGZsq{}}\PYG{p}{)}\PYG{p}{;}\PYG{n}{plt}\PYG{o}{.}\PYG{n}{axis}\PYG{p}{(}\PYG{l+s+s1}{\PYGZsq{}}\PYG{l+s+s1}{off}\PYG{l+s+s1}{\PYGZsq{}}\PYG{p}{)}\PYG{p}{;}
\PYG{n}{plt}\PYG{o}{.}\PYG{n}{subplot}\PYG{p}{(}\PYG{l+m+mi}{1}\PYG{p}{,}\PYG{l+m+mi}{3}\PYG{p}{,}\PYG{l+m+mi}{2}\PYG{p}{)}\PYG{p}{;}\PYG{n}{plt}\PYG{o}{.}\PYG{n}{imshow}\PYG{p}{(}\PYG{l+m+mf}{0.90}\PYG{o}{\PYGZlt{}}\PYG{n}{np}\PYG{o}{.}\PYG{n}{random}\PYG{o}{.}\PYG{n}{uniform}\PYG{p}{(}\PYG{l+m+mi}{0}\PYG{p}{,}\PYG{l+m+mi}{1}\PYG{p}{,}\PYG{n}{size}\PYG{o}{=}\PYG{p}{[}\PYG{l+m+mi}{100}\PYG{p}{,}\PYG{l+m+mi}{100}\PYG{p}{]}\PYG{p}{)}\PYG{p}{,}\PYG{n}{cmap}\PYG{o}{=}\PYG{l+s+s1}{\PYGZsq{}}\PYG{l+s+s1}{gray}\PYG{l+s+s1}{\PYGZsq{}}\PYG{p}{)}\PYG{p}{;} \PYG{n}{plt}\PYG{o}{.}\PYG{n}{title}\PYG{p}{(}\PYG{l+s+s2}{\PYGZdq{}}\PYG{l+s+s2}{Salt and pepper}\PYG{l+s+s2}{\PYGZdq{}}\PYG{p}{)}\PYG{p}{,}\PYG{n}{plt}\PYG{o}{.}\PYG{n}{axis}\PYG{p}{(}\PYG{l+s+s1}{\PYGZsq{}}\PYG{l+s+s1}{off}\PYG{l+s+s1}{\PYGZsq{}}\PYG{p}{)}\PYG{p}{;}
\PYG{n}{plt}\PYG{o}{.}\PYG{n}{subplot}\PYG{p}{(}\PYG{l+m+mi}{1}\PYG{p}{,}\PYG{l+m+mi}{3}\PYG{p}{,}\PYG{l+m+mi}{3}\PYG{p}{)}\PYG{p}{;}\PYG{n}{plt}\PYG{o}{.}\PYG{n}{imshow}\PYG{p}{(}\PYG{n}{ski}\PYG{o}{.}\PYG{n}{filters}\PYG{o}{.}\PYG{n}{gaussian}\PYG{p}{(}\PYG{n}{np}\PYG{o}{.}\PYG{n}{random}\PYG{o}{.}\PYG{n}{normal}\PYG{p}{(}\PYG{l+m+mi}{0}\PYG{p}{,}\PYG{l+m+mi}{1}\PYG{p}{,}\PYG{n}{size}\PYG{o}{=}\PYG{p}{[}\PYG{l+m+mi}{100}\PYG{p}{,}\PYG{l+m+mi}{100}\PYG{p}{]}\PYG{p}{)}\PYG{p}{,}\PYG{n}{sigma}\PYG{o}{=}\PYG{l+m+mi}{1}\PYG{p}{)}\PYG{p}{,}\PYG{n}{cmap}\PYG{o}{=}\PYG{l+s+s1}{\PYGZsq{}}\PYG{l+s+s1}{gray}\PYG{l+s+s1}{\PYGZsq{}}\PYG{p}{)}\PYG{p}{;} \PYG{n}{plt}\PYG{o}{.}\PYG{n}{title}\PYG{p}{(}\PYG{l+s+s2}{\PYGZdq{}}\PYG{l+s+s2}{Structured}\PYG{l+s+s2}{\PYGZdq{}}\PYG{p}{)}\PYG{p}{,}\PYG{n}{plt}\PYG{o}{.}\PYG{n}{axis}\PYG{p}{(}\PYG{l+s+s1}{\PYGZsq{}}\PYG{l+s+s1}{off}\PYG{l+s+s1}{\PYGZsq{}}\PYG{p}{)}\PYG{p}{;}
\PYG{n}{plt}\PYG{o}{.}\PYG{n}{tight\PYGZus{}layout}\PYG{p}{(}\PYG{p}{)}\PYG{p}{;}
\end{sphinxVerbatim}

\noindent\sphinxincludegraphics{{02-ImageEnhancement_24_0}.png}


\subsection{Noise models \sphinxhyphen{} Gaussian noise}
\label{\detokenize{02-ImageEnhancement:noise-models-gaussian-noise}}
\sphinxAtStartPar
Gaussian noise is the most common random distribution used. All other distributions asymptotically converges towards the Gaussian distribution thanks to the \sphinxhref{https://en.wikipedia.org/wiki/Central\_limit\_theorem}{central limit theorem}. The Gaussian noise is an easy distribution to work with when you derive signal processing models. This is also the reason why it is so popular to use this model also for non\sphinxhyphen{}Gaussian noise.
\begin{itemize}
\item {} 
\sphinxAtStartPar
Additive

\item {} 
\sphinxAtStartPar
Easy to model

\item {} 
\sphinxAtStartPar
Law of large numbers

\end{itemize}

\sphinxAtStartPar
\sphinxstylestrong{Distribution function}
\begin{equation*}
\begin{split}n(x)=\frac{1}{\sqrt{2\pi\sigma}}\exp{-\left(\frac{x-\mu}{2\sigma}\right)^2}\end{split}
\end{equation*}
\sphinxAtStartPar
Below you see plots of the Gaussian distribution with different parameters.

\begin{sphinxVerbatim}[commandchars=\\\{\}]
\PYG{k+kn}{from} \PYG{n+nn}{scipy}\PYG{n+nn}{.}\PYG{n+nn}{stats} \PYG{k+kn}{import} \PYG{n}{norm}
\PYG{n}{rv} \PYG{o}{=} \PYG{n}{norm}\PYG{p}{(}\PYG{n}{loc} \PYG{o}{=} \PYG{o}{\PYGZhy{}}\PYG{l+m+mf}{1.}\PYG{p}{,} \PYG{n}{scale} \PYG{o}{=} \PYG{l+m+mf}{1.0}\PYG{p}{)}\PYG{p}{;}\PYG{n}{rv1} \PYG{o}{=} \PYG{n}{norm}\PYG{p}{(}\PYG{n}{loc} \PYG{o}{=} \PYG{l+m+mf}{0.}\PYG{p}{,} \PYG{n}{scale} \PYG{o}{=} \PYG{l+m+mf}{2.0}\PYG{p}{)}\PYG{p}{;} \PYG{n}{rv2} \PYG{o}{=} \PYG{n}{norm}\PYG{p}{(}\PYG{n}{loc} \PYG{o}{=} \PYG{l+m+mf}{2.}\PYG{p}{,} \PYG{n}{scale} \PYG{o}{=} \PYG{l+m+mf}{3.0}\PYG{p}{)}
\PYG{n}{x} \PYG{o}{=} \PYG{n}{np}\PYG{o}{.}\PYG{n}{arange}\PYG{p}{(}\PYG{o}{\PYGZhy{}}\PYG{l+m+mi}{10}\PYG{p}{,} \PYG{l+m+mi}{10}\PYG{p}{,} \PYG{o}{.}\PYG{l+m+mi}{1}\PYG{p}{)}
\PYG{n}{plt}\PYG{o}{.}\PYG{n}{figure}\PYG{p}{(}\PYG{n}{figsize}\PYG{o}{=}\PYG{p}{(}\PYG{l+m+mi}{5}\PYG{p}{,}\PYG{l+m+mi}{4}\PYG{p}{)}\PYG{p}{)}
\PYG{c+c1}{\PYGZsh{}plot the pdfs of these normal distributions }
\PYG{n}{plt}\PYG{o}{.}\PYG{n}{plot}\PYG{p}{(}\PYG{n}{x}\PYG{p}{,} \PYG{n}{rv}\PYG{o}{.}\PYG{n}{pdf}\PYG{p}{(}\PYG{n}{x}\PYG{p}{)}\PYG{p}{,}\PYG{n}{label}\PYG{o}{=}\PYG{l+s+s1}{\PYGZsq{}}\PYG{l+s+s1}{\PYGZdl{}}\PYG{l+s+s1}{\PYGZbs{}}\PYG{l+s+s1}{mu\PYGZdl{}=\PYGZhy{}1, \PYGZdl{}}\PYG{l+s+s1}{\PYGZbs{}}\PYG{l+s+s1}{sigma\PYGZdl{}=1}\PYG{l+s+s1}{\PYGZsq{}}\PYG{p}{)}
\PYG{n}{plt}\PYG{o}{.}\PYG{n}{plot}\PYG{p}{(}\PYG{n}{x}\PYG{p}{,} \PYG{n}{rv1}\PYG{o}{.}\PYG{n}{pdf}\PYG{p}{(}\PYG{n}{x}\PYG{p}{)}\PYG{p}{,}\PYG{n}{label}\PYG{o}{=}\PYG{l+s+s1}{\PYGZsq{}}\PYG{l+s+s1}{\PYGZdl{}}\PYG{l+s+s1}{\PYGZbs{}}\PYG{l+s+s1}{mu\PYGZdl{}=0, \PYGZdl{}}\PYG{l+s+s1}{\PYGZbs{}}\PYG{l+s+s1}{sigma\PYGZdl{}=2}\PYG{l+s+s1}{\PYGZsq{}}\PYG{p}{)} 
\PYG{n}{plt}\PYG{o}{.}\PYG{n}{plot}\PYG{p}{(}\PYG{n}{x}\PYG{p}{,} \PYG{n}{rv2}\PYG{o}{.}\PYG{n}{pdf}\PYG{p}{(}\PYG{n}{x}\PYG{p}{)}\PYG{p}{,}\PYG{n}{label}\PYG{o}{=}\PYG{l+s+s1}{\PYGZsq{}}\PYG{l+s+s1}{\PYGZdl{}}\PYG{l+s+s1}{\PYGZbs{}}\PYG{l+s+s1}{mu\PYGZdl{}=2, \PYGZdl{}}\PYG{l+s+s1}{\PYGZbs{}}\PYG{l+s+s1}{sigma\PYGZdl{}=3}\PYG{l+s+s1}{\PYGZsq{}}\PYG{p}{)}
\PYG{n}{plt}\PYG{o}{.}\PYG{n}{legend}\PYG{p}{(}\PYG{p}{)}\PYG{p}{;}
\end{sphinxVerbatim}

\noindent\sphinxincludegraphics{{02-ImageEnhancement_29_0}.png}


\subsection{Noise models \sphinxhyphen{} Poisson noise}
\label{\detokenize{02-ImageEnhancement:noise-models-poisson-noise}}
\sphinxAtStartPar
The Poisson noise is the central noise model for event counting processes. It is thus the type of noise you see in imaging as the detectors in some sense is counting the number of particles arriving at the detector, e.g. photons or neutrons. This noise distribution changes shape with increasing number of particles; the distribution is clearly asymmetric for few particles while it takes a Gaussian shape when many particles are counted. It is also multiplicative in contrast to the Gaussian noise. This is in other words the noise distribution you need to use if you want to model image noise correctly.
\begin{itemize}
\item {} 
\sphinxAtStartPar
Multiplicative

\item {} 
\sphinxAtStartPar
Physically correct for event counting

\end{itemize}

\sphinxAtStartPar
\sphinxstylestrong{Distribition function}
\begin{equation*}
\begin{split}p(x)=\frac{\lambda^{k}}{k!} e^{-\lambda\,x}\end{split}
\end{equation*}
\sphinxAtStartPar
The plot below show a poisson distribtion for \(\lambda\)=3

\begin{sphinxVerbatim}[commandchars=\\\{\}]
\PYG{k+kn}{from} \PYG{n+nn}{scipy}\PYG{n+nn}{.}\PYG{n+nn}{stats} \PYG{k+kn}{import} \PYG{n}{poisson}
\PYG{n}{mu}\PYG{o}{=}\PYG{l+m+mi}{3}
\PYG{n}{fig}\PYG{p}{,} \PYG{n}{ax} \PYG{o}{=} \PYG{n}{plt}\PYG{o}{.}\PYG{n}{subplots}\PYG{p}{(}\PYG{l+m+mi}{1}\PYG{p}{,} \PYG{l+m+mi}{1}\PYG{p}{)}
\PYG{n}{x} \PYG{o}{=} \PYG{n}{np}\PYG{o}{.}\PYG{n}{arange}\PYG{p}{(}\PYG{n}{poisson}\PYG{o}{.}\PYG{n}{ppf}\PYG{p}{(}\PYG{l+m+mf}{0.01}\PYG{p}{,} \PYG{n}{mu}\PYG{p}{)}\PYG{p}{,}              \PYG{n}{poisson}\PYG{o}{.}\PYG{n}{ppf}\PYG{p}{(}\PYG{l+m+mf}{0.999}\PYG{p}{,} \PYG{n}{mu}\PYG{p}{)}\PYG{p}{)}
\PYG{n}{ax}\PYG{o}{.}\PYG{n}{plot}\PYG{p}{(}\PYG{n}{x}\PYG{p}{,} \PYG{n}{poisson}\PYG{o}{.}\PYG{n}{pmf}\PYG{p}{(}\PYG{n}{x}\PYG{p}{,} \PYG{n}{mu}\PYG{p}{)}\PYG{p}{,} \PYG{l+s+s1}{\PYGZsq{}}\PYG{l+s+s1}{bo}\PYG{l+s+s1}{\PYGZsq{}}\PYG{p}{,} \PYG{n}{ms}\PYG{o}{=}\PYG{l+m+mi}{8}\PYG{p}{,} \PYG{n}{label}\PYG{o}{=}\PYG{l+s+s1}{\PYGZsq{}}\PYG{l+s+s1}{poisson pmf}\PYG{l+s+s1}{\PYGZsq{}}\PYG{p}{)}
\PYG{n}{ax}\PYG{o}{.}\PYG{n}{vlines}\PYG{p}{(}\PYG{n}{x}\PYG{p}{,} \PYG{l+m+mi}{0}\PYG{p}{,} \PYG{n}{poisson}\PYG{o}{.}\PYG{n}{pmf}\PYG{p}{(}\PYG{n}{x}\PYG{p}{,} \PYG{n}{mu}\PYG{p}{)}\PYG{p}{,} \PYG{n}{colors}\PYG{o}{=}\PYG{l+s+s1}{\PYGZsq{}}\PYG{l+s+s1}{b}\PYG{l+s+s1}{\PYGZsq{}}\PYG{p}{,} \PYG{n}{lw}\PYG{o}{=}\PYG{l+m+mi}{5}\PYG{p}{,} \PYG{n}{alpha}\PYG{o}{=}\PYG{l+m+mf}{0.5}\PYG{p}{)}\PYG{p}{;}
\end{sphinxVerbatim}

\noindent\sphinxincludegraphics{{02-ImageEnhancement_34_0}.png}


\subsection{Compare Gaussian and Possion noise}
\label{\detokenize{02-ImageEnhancement:compare-gaussian-and-possion-noise}}
\sphinxAtStartPar
Now, let’s compare relaizations of Gaussian and Poisson noise overlaid on a sine curve. The important thing to observe is that the noise amplitude is independent of the signal value and constant for the Gaussian noise. For Poisson noise it is very different. The noise amplitude changes with the signal values. Higher signal strength also produces greater noise amplitudes.

\begin{sphinxVerbatim}[commandchars=\\\{\}]
\PYG{n}{plt}\PYG{o}{.}\PYG{n}{figure}\PYG{p}{(}\PYG{n}{figsize}\PYG{o}{=}\PYG{p}{[}\PYG{l+m+mi}{15}\PYG{p}{,}\PYG{l+m+mi}{8}\PYG{p}{]}\PYG{p}{)}
\PYG{n}{x}\PYG{o}{=}\PYG{n}{np}\PYG{o}{.}\PYG{n}{linspace}\PYG{p}{(}\PYG{l+m+mi}{0}\PYG{p}{,}\PYG{l+m+mi}{2}\PYG{o}{*}\PYG{n}{np}\PYG{o}{.}\PYG{n}{pi}\PYG{p}{,}\PYG{l+m+mi}{100}\PYG{p}{)}\PYG{p}{;} 
\PYG{n}{y}\PYG{o}{=}\PYG{l+m+mi}{10}\PYG{o}{*}\PYG{n}{np}\PYG{o}{.}\PYG{n}{sin}\PYG{p}{(}\PYG{n}{x}\PYG{p}{)}\PYG{o}{+}\PYG{l+m+mi}{11}\PYG{p}{;} \PYG{n}{ng}\PYG{o}{=}\PYG{n}{np}\PYG{o}{.}\PYG{n}{random}\PYG{o}{.}\PYG{n}{normal}\PYG{p}{(}\PYG{l+m+mi}{0}\PYG{p}{,}\PYG{l+m+mi}{1}\PYG{p}{,}\PYG{n}{size}\PYG{o}{=}\PYG{n+nb}{len}\PYG{p}{(}\PYG{n}{x}\PYG{p}{)}\PYG{p}{)}\PYG{p}{;} \PYG{n}{npoi} \PYG{o}{=} \PYG{n}{np}\PYG{o}{.}\PYG{n}{random}\PYG{o}{.}\PYG{n}{poisson}\PYG{p}{(}\PYG{n}{y}\PYG{p}{)}\PYG{p}{;}
\PYG{n}{plt}\PYG{o}{.}\PYG{n}{subplot}\PYG{p}{(}\PYG{l+m+mi}{2}\PYG{p}{,}\PYG{l+m+mi}{2}\PYG{p}{,}\PYG{l+m+mi}{1}\PYG{p}{)}\PYG{p}{;} \PYG{n}{plt}\PYG{o}{.}\PYG{n}{plot}\PYG{p}{(}\PYG{n}{x}\PYG{p}{,}\PYG{n}{y}\PYG{o}{+}\PYG{n}{ng}\PYG{p}{)}\PYG{p}{;}\PYG{n}{plt}\PYG{o}{.}\PYG{n}{plot}\PYG{p}{(}\PYG{n}{x}\PYG{p}{,}\PYG{n}{y}\PYG{p}{)}\PYG{p}{;} \PYG{n}{plt}\PYG{o}{.}\PYG{n}{axis}\PYG{p}{(}\PYG{l+s+s1}{\PYGZsq{}}\PYG{l+s+s1}{off}\PYG{l+s+s1}{\PYGZsq{}}\PYG{p}{)}\PYG{p}{;}\PYG{n}{plt}\PYG{o}{.}\PYG{n}{title}\PYG{p}{(}\PYG{l+s+s1}{\PYGZsq{}}\PYG{l+s+s1}{Gaussian}\PYG{l+s+s1}{\PYGZsq{}}\PYG{p}{)}\PYG{p}{;} \PYG{n}{plt}\PYG{o}{.}\PYG{n}{subplot}\PYG{p}{(}\PYG{l+m+mi}{2}\PYG{p}{,}\PYG{l+m+mi}{2}\PYG{p}{,}\PYG{l+m+mi}{3}\PYG{p}{)}\PYG{p}{;}\PYG{n}{plt}\PYG{o}{.}\PYG{n}{plot}\PYG{p}{(}\PYG{n}{x}\PYG{p}{,}\PYG{n}{ng}\PYG{p}{)}\PYG{p}{;}\PYG{n}{plt}\PYG{o}{.}\PYG{n}{axis}\PYG{p}{(}\PYG{l+s+s1}{\PYGZsq{}}\PYG{l+s+s1}{off}\PYG{l+s+s1}{\PYGZsq{}}\PYG{p}{)}\PYG{p}{;}
\PYG{n}{plt}\PYG{o}{.}\PYG{n}{subplot}\PYG{p}{(}\PYG{l+m+mi}{2}\PYG{p}{,}\PYG{l+m+mi}{2}\PYG{p}{,}\PYG{l+m+mi}{2}\PYG{p}{)}\PYG{p}{;} \PYG{n}{plt}\PYG{o}{.}\PYG{n}{plot}\PYG{p}{(}\PYG{n}{x}\PYG{p}{,}\PYG{n}{npoi}\PYG{p}{)}\PYG{p}{;}\PYG{n}{plt}\PYG{o}{.}\PYG{n}{plot}\PYG{p}{(}\PYG{n}{x}\PYG{p}{,}\PYG{n}{y}\PYG{p}{)}\PYG{p}{;} \PYG{n}{plt}\PYG{o}{.}\PYG{n}{axis}\PYG{p}{(}\PYG{l+s+s1}{\PYGZsq{}}\PYG{l+s+s1}{off}\PYG{l+s+s1}{\PYGZsq{}}\PYG{p}{)}\PYG{p}{;}\PYG{n}{plt}\PYG{o}{.}\PYG{n}{title}\PYG{p}{(}\PYG{l+s+s1}{\PYGZsq{}}\PYG{l+s+s1}{Poisson}\PYG{l+s+s1}{\PYGZsq{}}\PYG{p}{)}\PYG{p}{;} \PYG{n}{plt}\PYG{o}{.}\PYG{n}{subplot}\PYG{p}{(}\PYG{l+m+mi}{2}\PYG{p}{,}\PYG{l+m+mi}{2}\PYG{p}{,}\PYG{l+m+mi}{4}\PYG{p}{)}\PYG{p}{;}\PYG{n}{plt}\PYG{o}{.}\PYG{n}{plot}\PYG{p}{(}\PYG{n}{x}\PYG{p}{,}\PYG{n}{npoi}\PYG{o}{\PYGZhy{}}\PYG{n}{y}\PYG{p}{)}\PYG{p}{;}\PYG{n}{plt}\PYG{o}{.}\PYG{n}{axis}\PYG{p}{(}\PYG{l+s+s1}{\PYGZsq{}}\PYG{l+s+s1}{off}\PYG{l+s+s1}{\PYGZsq{}}\PYG{p}{)}\PYG{p}{;}
\PYG{n}{plt}\PYG{o}{.}\PYG{n}{suptitle}\PYG{p}{(}\PYG{l+s+s1}{\PYGZsq{}}\PYG{l+s+s1}{Samples of two distributions}\PYG{l+s+s1}{\PYGZsq{}}\PYG{p}{)}\PYG{p}{;}
\end{sphinxVerbatim}

\noindent\sphinxincludegraphics{{02-ImageEnhancement_37_0}.png}


\subsection{Noise models \sphinxhyphen{} Salt’n’pepper noise}
\label{\detokenize{02-ImageEnhancement:noise-models-salt-n-pepper-noise}}\begin{itemize}
\item {} 
\sphinxAtStartPar
A type of outlier noise

\item {} 
\sphinxAtStartPar
Noise frequency described as probability of outlier

\item {} 
\sphinxAtStartPar
Can be additive, multiplicative, and independent replacement

\end{itemize}

\sphinxAtStartPar
\sphinxstylestrong{Example model}
\$\(sp(x)=\left\{\begin{array}{ll}
-1 & x\leq\lambda_1\\ 
0 & \lambda_1< x \leq \lambda_2\\
1 & \lambda_2<x
\end{array}\right.\qquad \begin{array}{l}x\in\mathcal{U}(0,1)\\\lambda_1<\lambda_2\\
\lambda_1+\lambda_2 = \mbox{noise fraction}
\end{array}\)\$


\subsection{Salt’n’pepper examples}
\label{\detokenize{02-ImageEnhancement:salt-n-pepper-examples}}
\begin{sphinxVerbatim}[commandchars=\\\{\}]
\PYG{k}{def} \PYG{n+nf}{snp}\PYG{p}{(}\PYG{n}{dims}\PYG{p}{,}\PYG{n}{Pblack}\PYG{p}{,}\PYG{n}{Pwhite}\PYG{p}{)} \PYG{p}{:} \PYG{c+c1}{\PYGZsh{} Noise model function}
    \PYG{n}{uni}\PYG{o}{=}\PYG{n}{np}\PYG{o}{.}\PYG{n}{random}\PYG{o}{.}\PYG{n}{uniform}\PYG{p}{(}\PYG{l+m+mi}{0}\PYG{p}{,}\PYG{l+m+mi}{1}\PYG{p}{,}\PYG{n}{dims}\PYG{p}{)}
    \PYG{n}{img}\PYG{o}{=}\PYG{p}{(}\PYG{n}{Pwhite}\PYG{o}{\PYGZlt{}}\PYG{n}{uni}\PYG{p}{)}\PYG{o}{.}\PYG{n}{astype}\PYG{p}{(}\PYG{n+nb}{float}\PYG{p}{)}\PYG{o}{\PYGZhy{}}\PYG{p}{(}\PYG{n}{uni}\PYG{o}{\PYGZlt{}}\PYG{n}{Pblack}\PYG{p}{)}\PYG{o}{.}\PYG{n}{astype}\PYG{p}{(}\PYG{n+nb}{float}\PYG{p}{)}
    \PYG{k}{return} \PYG{n}{img}
\end{sphinxVerbatim}

\begin{sphinxVerbatim}[commandchars=\\\{\}]
\PYG{n}{img10\PYGZus{}90}\PYG{o}{=}\PYG{n}{snp}\PYG{p}{(}\PYG{p}{[}\PYG{l+m+mi}{100}\PYG{p}{,}\PYG{l+m+mi}{100}\PYG{p}{]}\PYG{p}{,}\PYG{l+m+mf}{0.1}\PYG{p}{,}\PYG{l+m+mf}{0.9}\PYG{p}{)}\PYG{p}{;} \PYG{n}{img5\PYGZus{}95}\PYG{o}{=}\PYG{n}{snp}\PYG{p}{(}\PYG{p}{[}\PYG{l+m+mi}{100}\PYG{p}{,}\PYG{l+m+mi}{100}\PYG{p}{]}\PYG{p}{,}\PYG{l+m+mf}{0.05}\PYG{p}{,}\PYG{l+m+mf}{0.95}\PYG{p}{)}\PYG{p}{;}\PYG{n}{img1\PYGZus{}99}\PYG{o}{=}\PYG{n}{snp}\PYG{p}{(}\PYG{p}{[}\PYG{l+m+mi}{100}\PYG{p}{,}\PYG{l+m+mi}{100}\PYG{p}{]}\PYG{p}{,}\PYG{l+m+mf}{0.01}\PYG{p}{,}\PYG{l+m+mf}{0.99}\PYG{p}{)}
\PYG{n}{plt}\PYG{o}{.}\PYG{n}{figure}\PYG{p}{(}\PYG{n}{figsize}\PYG{o}{=}\PYG{p}{[}\PYG{l+m+mi}{15}\PYG{p}{,}\PYG{l+m+mi}{5}\PYG{p}{]}\PYG{p}{)}
\PYG{n}{plt}\PYG{o}{.}\PYG{n}{subplot}\PYG{p}{(}\PYG{l+m+mi}{1}\PYG{p}{,}\PYG{l+m+mi}{3}\PYG{p}{,}\PYG{l+m+mi}{1}\PYG{p}{)}\PYG{p}{;} \PYG{n}{plt}\PYG{o}{.}\PYG{n}{imshow}\PYG{p}{(}\PYG{n}{img1\PYGZus{}99}\PYG{p}{,}\PYG{n}{cmap}\PYG{o}{=}\PYG{l+s+s1}{\PYGZsq{}}\PYG{l+s+s1}{gray}\PYG{l+s+s1}{\PYGZsq{}}\PYG{p}{)}\PYG{p}{;} \PYG{n}{plt}\PYG{o}{.}\PYG{n}{title}\PYG{p}{(}\PYG{l+s+s1}{\PYGZsq{}}\PYG{l+s+s1}{\PYGZdl{}}\PYG{l+s+s1}{\PYGZbs{}}\PYG{l+s+s1}{lambda\PYGZus{}1\PYGZdl{}=1}\PYG{l+s+si}{\PYGZpc{} a}\PYG{l+s+s1}{nd \PYGZdl{}}\PYG{l+s+s1}{\PYGZbs{}}\PYG{l+s+s1}{lambda\PYGZus{}2\PYGZdl{}=99}\PYG{l+s+s1}{\PYGZpc{}}\PYG{l+s+s1}{\PYGZsq{}}\PYG{p}{,}\PYG{n}{fontsize}\PYG{o}{=}\PYG{l+m+mi}{16}\PYG{p}{)}\PYG{p}{;} \PYG{n}{plt}\PYG{o}{.}\PYG{n}{axis}\PYG{p}{(}\PYG{l+s+s1}{\PYGZsq{}}\PYG{l+s+s1}{off}\PYG{l+s+s1}{\PYGZsq{}}\PYG{p}{)}\PYG{p}{;}
\PYG{n}{plt}\PYG{o}{.}\PYG{n}{subplot}\PYG{p}{(}\PYG{l+m+mi}{1}\PYG{p}{,}\PYG{l+m+mi}{3}\PYG{p}{,}\PYG{l+m+mi}{2}\PYG{p}{)}\PYG{p}{;} \PYG{n}{plt}\PYG{o}{.}\PYG{n}{imshow}\PYG{p}{(}\PYG{n}{img5\PYGZus{}95}\PYG{p}{,}\PYG{n}{cmap}\PYG{o}{=}\PYG{l+s+s1}{\PYGZsq{}}\PYG{l+s+s1}{gray}\PYG{l+s+s1}{\PYGZsq{}}\PYG{p}{)}\PYG{p}{;} \PYG{n}{plt}\PYG{o}{.}\PYG{n}{title}\PYG{p}{(}\PYG{l+s+s1}{\PYGZsq{}}\PYG{l+s+s1}{\PYGZdl{}}\PYG{l+s+s1}{\PYGZbs{}}\PYG{l+s+s1}{lambda\PYGZus{}1\PYGZdl{}=5}\PYG{l+s+si}{\PYGZpc{} a}\PYG{l+s+s1}{nd \PYGZdl{}}\PYG{l+s+s1}{\PYGZbs{}}\PYG{l+s+s1}{lambda\PYGZus{}2\PYGZdl{}=95}\PYG{l+s+s1}{\PYGZpc{}}\PYG{l+s+s1}{\PYGZsq{}}\PYG{p}{,}\PYG{n}{fontsize}\PYG{o}{=}\PYG{l+m+mi}{16}\PYG{p}{)}\PYG{p}{;} \PYG{n}{plt}\PYG{o}{.}\PYG{n}{axis}\PYG{p}{(}\PYG{l+s+s1}{\PYGZsq{}}\PYG{l+s+s1}{off}\PYG{l+s+s1}{\PYGZsq{}}\PYG{p}{)}\PYG{p}{;}
\PYG{n}{plt}\PYG{o}{.}\PYG{n}{subplot}\PYG{p}{(}\PYG{l+m+mi}{1}\PYG{p}{,}\PYG{l+m+mi}{3}\PYG{p}{,}\PYG{l+m+mi}{3}\PYG{p}{)}\PYG{p}{;} \PYG{n}{plt}\PYG{o}{.}\PYG{n}{imshow}\PYG{p}{(}\PYG{n}{img10\PYGZus{}90}\PYG{p}{,}\PYG{n}{cmap}\PYG{o}{=}\PYG{l+s+s1}{\PYGZsq{}}\PYG{l+s+s1}{gray}\PYG{l+s+s1}{\PYGZsq{}}\PYG{p}{)}\PYG{p}{;} \PYG{n}{plt}\PYG{o}{.}\PYG{n}{title}\PYG{p}{(}\PYG{l+s+s1}{\PYGZsq{}}\PYG{l+s+s1}{\PYGZdl{}}\PYG{l+s+s1}{\PYGZbs{}}\PYG{l+s+s1}{lambda\PYGZus{}1\PYGZdl{}=10}\PYG{l+s+si}{\PYGZpc{} a}\PYG{l+s+s1}{nd \PYGZdl{}}\PYG{l+s+s1}{\PYGZbs{}}\PYG{l+s+s1}{lambda\PYGZus{}2\PYGZdl{}=90}\PYG{l+s+s1}{\PYGZpc{}}\PYG{l+s+s1}{\PYGZsq{}}\PYG{p}{,}\PYG{n}{fontsize}\PYG{o}{=}\PYG{l+m+mi}{16}\PYG{p}{)}\PYG{p}{;} \PYG{n}{plt}\PYG{o}{.}\PYG{n}{axis}\PYG{p}{(}\PYG{l+s+s1}{\PYGZsq{}}\PYG{l+s+s1}{off}\PYG{l+s+s1}{\PYGZsq{}}\PYG{p}{)}\PYG{p}{;}
\end{sphinxVerbatim}

\noindent\sphinxincludegraphics{{02-ImageEnhancement_41_0}.png}


\section{Signal to noise ratio}
\label{\detokenize{02-ImageEnhancement:signal-to-noise-ratio}}
\sphinxAtStartPar
It is important to know how strong the noise is compared to the signal in order to decide how to proceed with the analysis. Therefore, we need a metric to quantify the noise.

\sphinxAtStartPar
The Signal to noise ratio measures the noise strengh in a signal

\sphinxAtStartPar
\sphinxstylestrong{Definition}
\begin{equation*}
\begin{split}SNR=\frac{mean(f)}{stddev(f)}\end{split}
\end{equation*}
\sphinxAtStartPar
Sometimes the term contrast to noise ratio is also used. This means that you measure the intensity difference in between two relevant features and divide this by the noise.


\subsection{Signal to noise ratio for Poisson noise}
\label{\detokenize{02-ImageEnhancement:signal-to-noise-ratio-for-poisson-noise}}
\sphinxAtStartPar
The SNR of poisson noise is particularly easy to compute because \(E[x]=v[x]\). This means that the SNR is proportional to the square root of the number of particles.
\begin{itemize}
\item {} 
\sphinxAtStartPar
For a Poisson distribution the SNR is :

\end{itemize}
\begin{equation*}
\begin{split}SNR=\frac{E[x]}{s[x]}\sim\frac{N}{\sqrt{N}}=\sqrt{N}\end{split}
\end{equation*}\begin{itemize}
\item {} 
\sphinxAtStartPar
\(N\) is the number of particles \(\sim\) exposure time

\end{itemize}

\sphinxAtStartPar
where \sphinxstyleemphasis{N} is the number of captured particles. The figure below shows two neutron images acquired at 0.1s and 10s respectively. The plot shows the signal to noise ratio obtained for different exposure times.

\sphinxAtStartPar
The signal to noise ratio can be improved by increasing the number of neutrons per pixel. This can be achived through increasing
\begin{itemize}
\item {} 
\sphinxAtStartPar
Neutron flux \sphinxhyphen{} this is usually relatively hard as the neutron sources operate with the parameters it is designed for. There is a posibilty by changing the neutron aperture, but has an impact of the beam quality.

\item {} 
\sphinxAtStartPar
Exposure time \sphinxhyphen{} the exposure time can be increased but in the end there is a limitation on how much this can be used. Beam time is limited which means the experiment must be finished in a given time. There is also an upper limit on the exposure time defined by the observed sample or process when it changes over time. Too long exposure times will result in motion artefacts.

\item {} 
\sphinxAtStartPar
Pixel size \sphinxhyphen{} increasing the pixel size means that neutrons are collected over a greater area and thus more neutrons are captured during the exposure. The limit on how much you can increase the pixel size is defined by the smallest features you want to detect.

\item {} 
\sphinxAtStartPar
Detector material and thickness \sphinxhyphen{} the number of captured neutrons depends on the scintillator material and how thick it is. The thickness does however have an impact on the resolution. Therefore scintillator thickness and pixel size often increase in parallel as there is no point in oversampling a smooth signal to much.

\end{itemize}

\sphinxAtStartPar
In the end, there are many parameters that combined results in the SNR you obtain. These parameters are tuned to match the experiment conditions. The filtering techniques presented in this lecture can help to increase the SNR and hopefuly make the way for a quantitaive analysis.

\begin{sphinxVerbatim}[commandchars=\\\{\}]
\PYG{n}{exptime}\PYG{o}{=}\PYG{n}{np}\PYG{o}{.}\PYG{n}{array}\PYG{p}{(}\PYG{p}{[}\PYG{l+m+mi}{50}\PYG{p}{,}\PYG{l+m+mi}{100}\PYG{p}{,}\PYG{l+m+mi}{200}\PYG{p}{,}\PYG{l+m+mi}{500}\PYG{p}{,}\PYG{l+m+mi}{1000}\PYG{p}{,}\PYG{l+m+mi}{2000}\PYG{p}{,}\PYG{l+m+mi}{5000}\PYG{p}{,}\PYG{l+m+mi}{10000}\PYG{p}{]}\PYG{p}{)}
\PYG{n}{snr} \PYG{o}{=} \PYG{n}{np}\PYG{o}{.}\PYG{n}{array}\PYG{p}{(}\PYG{p}{[} \PYG{l+m+mf}{8.45949767}\PYG{p}{,} \PYG{l+m+mf}{11.40011621}\PYG{p}{,} \PYG{l+m+mf}{16.38118766}\PYG{p}{,} \PYG{l+m+mf}{21.12056507}\PYG{p}{,} \PYG{l+m+mf}{31.09116641}\PYG{p}{,}\PYG{l+m+mf}{40.65323123}\PYG{p}{,} \PYG{l+m+mf}{55.60833117}\PYG{p}{,} \PYG{l+m+mf}{68.21108979}\PYG{p}{]}\PYG{p}{)}\PYG{p}{;}
\PYG{n}{marker\PYGZus{}style} \PYG{o}{=} \PYG{n+nb}{dict}\PYG{p}{(}\PYG{n}{color}\PYG{o}{=}\PYG{l+s+s1}{\PYGZsq{}}\PYG{l+s+s1}{cornflowerblue}\PYG{l+s+s1}{\PYGZsq{}}\PYG{p}{,} \PYG{n}{linestyle}\PYG{o}{=}\PYG{l+s+s1}{\PYGZsq{}}\PYG{l+s+s1}{\PYGZhy{}}\PYG{l+s+s1}{\PYGZsq{}}\PYG{p}{,} \PYG{n}{marker}\PYG{o}{=}\PYG{l+s+s1}{\PYGZsq{}}\PYG{l+s+s1}{o}\PYG{l+s+s1}{\PYGZsq{}}\PYG{p}{,}\PYG{n}{markersize}\PYG{o}{=}\PYG{l+m+mi}{10}\PYG{p}{,} \PYG{n}{markerfacecoloralt}\PYG{o}{=}\PYG{l+s+s1}{\PYGZsq{}}\PYG{l+s+s1}{gray}\PYG{l+s+s1}{\PYGZsq{}}\PYG{p}{)}\PYG{p}{;}
\PYG{n}{plt}\PYG{o}{.}\PYG{n}{figure}\PYG{p}{(}\PYG{n}{figsize}\PYG{o}{=}\PYG{p}{(}\PYG{l+m+mi}{15}\PYG{p}{,}\PYG{l+m+mi}{5}\PYG{p}{)}\PYG{p}{)} 
\PYG{n}{plt}\PYG{o}{.}\PYG{n}{subplot}\PYG{p}{(}\PYG{l+m+mi}{1}\PYG{p}{,}\PYG{l+m+mi}{3}\PYG{p}{,}\PYG{l+m+mi}{2}\PYG{p}{)}\PYG{p}{;}\PYG{n}{plt}\PYG{o}{.}\PYG{n}{plot}\PYG{p}{(}\PYG{n}{exptime}\PYG{o}{/}\PYG{l+m+mi}{1000}\PYG{p}{,}\PYG{n}{snr}\PYG{p}{,} \PYG{o}{*}\PYG{o}{*}\PYG{n}{marker\PYGZus{}style}\PYG{p}{)}\PYG{p}{;}\PYG{n}{plt}\PYG{o}{.}\PYG{n}{xlabel}\PYG{p}{(}\PYG{l+s+s1}{\PYGZsq{}}\PYG{l+s+s1}{Exposure time [s]}\PYG{l+s+s1}{\PYGZsq{}}\PYG{p}{)}\PYG{p}{;}\PYG{n}{plt}\PYG{o}{.}\PYG{n}{ylabel}\PYG{p}{(}\PYG{l+s+s1}{\PYGZsq{}}\PYG{l+s+s1}{SNR [1]}\PYG{l+s+s1}{\PYGZsq{}}\PYG{p}{)}
\PYG{n}{img50ms}\PYG{o}{=}\PYG{n}{plt}\PYG{o}{.}\PYG{n}{imread}\PYG{p}{(}\PYG{l+s+s1}{\PYGZsq{}}\PYG{l+s+s1}{figures//tower\PYGZus{}50ms.png}\PYG{l+s+s1}{\PYGZsq{}}\PYG{p}{)}\PYG{p}{;} \PYG{n}{img10000ms}\PYG{o}{=}\PYG{n}{plt}\PYG{o}{.}\PYG{n}{imread}\PYG{p}{(}\PYG{l+s+s1}{\PYGZsq{}}\PYG{l+s+s1}{figures/tower\PYGZus{}10000ms.png}\PYG{l+s+s1}{\PYGZsq{}}\PYG{p}{)}\PYG{p}{;}
\PYG{n}{plt}\PYG{o}{.}\PYG{n}{subplot}\PYG{p}{(}\PYG{l+m+mi}{1}\PYG{p}{,}\PYG{l+m+mi}{3}\PYG{p}{,}\PYG{l+m+mi}{1}\PYG{p}{)}\PYG{p}{;}\PYG{n}{plt}\PYG{o}{.}\PYG{n}{imshow}\PYG{p}{(}\PYG{n}{img50ms}\PYG{p}{)}\PYG{p}{;} \PYG{n}{plt}\PYG{o}{.}\PYG{n}{title}\PYG{p}{(}\PYG{l+s+s1}{\PYGZsq{}}\PYG{l+s+s1}{50ms}\PYG{l+s+s1}{\PYGZsq{}}\PYG{p}{)}\PYG{p}{;} \PYG{n}{plt}\PYG{o}{.}\PYG{n}{subplot}\PYG{p}{(}\PYG{l+m+mi}{1}\PYG{p}{,}\PYG{l+m+mi}{3}\PYG{p}{,}\PYG{l+m+mi}{3}\PYG{p}{)}\PYG{p}{;} \PYG{n}{plt}\PYG{o}{.}\PYG{n}{imshow}\PYG{p}{(}\PYG{n}{img10000ms}\PYG{p}{)}\PYG{p}{,}\PYG{n}{plt}\PYG{o}{.}\PYG{n}{title}\PYG{p}{(}\PYG{l+s+s1}{\PYGZsq{}}\PYG{l+s+s1}{10s}\PYG{l+s+s1}{\PYGZsq{}}\PYG{p}{)}\PYG{p}{;}
\end{sphinxVerbatim}

\noindent\sphinxincludegraphics{{02-ImageEnhancement_50_0}.png}


\section{Useful python functions}
\label{\detokenize{02-ImageEnhancement:useful-python-functions}}

\subsection{Random number generators {[}numpy.random{]}}
\label{\detokenize{02-ImageEnhancement:random-number-generators-numpy-random}}
\sphinxAtStartPar
Generate an \(m \times n\) random fields with different distributions:
\begin{itemize}
\item {} 
\sphinxAtStartPar
\sphinxstylestrong{Gauss} \sphinxcode{\sphinxupquote{np.random.normal(mu,sigma, size={[}rows,cols{]})}}

\item {} 
\sphinxAtStartPar
\sphinxstylestrong{Uniform} \sphinxcode{\sphinxupquote{np.random.uniform(low,high,size={[}rows,cols{]})}}

\item {} 
\sphinxAtStartPar
\sphinxstylestrong{Poisson} \sphinxcode{\sphinxupquote{np.random.poisson(lambda, size={[}rows,cols{]})}}

\end{itemize}


\subsection{Statistics}
\label{\detokenize{02-ImageEnhancement:statistics}}\begin{itemize}
\item {} 
\sphinxAtStartPar
\sphinxcode{\sphinxupquote{np.mean(f)}}, \sphinxcode{\sphinxupquote{np.var(f)}}, \sphinxcode{\sphinxupquote{np.std(f)}} Computes the mean, variance, and standard deviation of an image \(f\).

\item {} 
\sphinxAtStartPar
\sphinxcode{\sphinxupquote{np.min(f)}},\sphinxcode{\sphinxupquote{np.max(f)}} Finds minimum and maximum values in \(f\).

\item {} 
\sphinxAtStartPar
\sphinxcode{\sphinxupquote{np.median(f)}}, \sphinxcode{\sphinxupquote{np.rank()}} Selects different values from the sorted data.

\end{itemize}


\chapter{Basic filtering}
\label{\detokenize{02-ImageEnhancement:basic-filtering}}

\section{What is a filter?}
\label{\detokenize{02-ImageEnhancement:what-is-a-filter}}
\sphinxAtStartPar
In general terms a filter is a component that separates mixed components form each other. In chemistry you use a filter to separate solid patricles from liquid. In signal processing the filter is used to separate frequencies in signals. In image processing we are not only talking about frequencies but also structures. This is something we will look into in a few weeks when we talk about morphological image processing.


\subsection{General definition}
\label{\detokenize{02-ImageEnhancement:general-definition}}
\sphinxAtStartPar
A filter is a processing unit that
\begin{itemize}
\item {} 
\sphinxAtStartPar
Enhances the wanted information

\item {} 
\sphinxAtStartPar
Suppresses the unwanted information

\end{itemize}



\sphinxAtStartPar
The filter should ideally perform the task it is meant to do, but at the same time maintain the information we want to keep. This is often a difficult problem, in particular with traditional filters. They apply the same characteristics to any pixel without concerning what this pixel actually represents. It only sees frequencies! As a consequence you may cancel all relevant information in the image in your mission to remove the noise.


\section{Filter characteristics}
\label{\detokenize{02-ImageEnhancement:filter-characteristics}}
\sphinxAtStartPar
Filters are characterized by the type of information they suppress or amplify.

\sphinxAtStartPar
In signal and image processing filters are applied to modify the amplitudes of different frequencies in the signal. Slow variations have low frequencies while rapid variations have high frequencies.


\subsection{Low\sphinxhyphen{}pass filters}
\label{\detokenize{02-ImageEnhancement:low-pass-filters}}
\sphinxAtStartPar
Low pass filters are designed to suppress frequencies in the upper part of the spectrum in order to better show slow changes in the images. The effect of a lowpass filter is that the images are blurred.

\begin{figure}[htbp]
\centering
\capstart

\noindent\sphinxincludegraphics[scale=0.5]{{lp_principle}.pdf}
\caption{The principle of a low\sphinxhyphen{}pass filter.}\label{\detokenize{02-ImageEnhancement:id6}}\end{figure}
\begin{itemize}
\item {} 
\sphinxAtStartPar
Slow changes are enhanced

\item {} 
\sphinxAtStartPar
Rapid changes are suppressed

\end{itemize}




\subsection{High\sphinxhyphen{}pass filters}
\label{\detokenize{02-ImageEnhancement:high-pass-filters}}
\sphinxAtStartPar
High pass filters are the opposite of the low pass filters as the name suggests. The suppress low frequency components of the spectrum leaving the rapid changes untouched. This is an important filter for edge detection as we will see later.

\begin{figure}[htbp]
\centering
\capstart

\noindent\sphinxincludegraphics[scale=0.5]{{hp_principle}.pdf}
\caption{The principle of a high\sphinxhyphen{}pass filter.}\label{\detokenize{02-ImageEnhancement:id7}}\end{figure}
\begin{itemize}
\item {} 
\sphinxAtStartPar
Rapid changes are enhanced

\item {} 
\sphinxAtStartPar
Slow changes are suppressed

\end{itemize}




\chapter{Basic filters}
\label{\detokenize{02-ImageEnhancement:basic-filters}}

\section{Linear filters}
\label{\detokenize{02-ImageEnhancement:linear-filters}}
\sphinxAtStartPar
Computed using the convolution operation
\begin{equation*}
\begin{split}g(x)=h*f(x)=\int_{\Omega}f(x-\tau) h(\tau) d\tau\end{split}
\end{equation*}
\sphinxAtStartPar
where
\begin{itemize}
\item {} 
\sphinxAtStartPar
\sphinxstylestrong{\(f\)} is the image

\item {} 
\sphinxAtStartPar
\sphinxstylestrong{\(h\)} is the convolution kernel of the filter

\end{itemize}




\section{Low\sphinxhyphen{}pass filter kernels}
\label{\detokenize{02-ImageEnhancement:low-pass-filter-kernels}}
\sphinxAtStartPar
The most common linear filters used in image processing are the box filter and the Gauss filter. The box filter has a kernel where all filter weights have the same strength. This filter essentially computes the local average of the neighborhood it covers.
A 5x5 box filter looks like this
\begin{equation*}
\begin{split}B=\frac{1}{25}\cdot\begin{array}{|c|c|c|c|c|}
\hline
1 & 1 & 1 & 1& 1\\
\hline
1 & 1 & 1 & 1& 1\\
\hline
1 & 1 & 1 & 1& 1\\
\hline
1 & 1 & 1 & 1& 1\\
\hline
1 & 1 & 1 & 1& 1\\
\hline
\end{array}
\end{split}
\end{equation*}
\sphinxAtStartPar
The scaling by the number of weights is sometimes omitted, but this would mean that the intensity of the resulting image is upscaled by this factor.

\sphinxAtStartPar
The Gauss filter kernel has its weights from the N\sphinxhyphen{}Dimensional Gauss function. Here, a 2D kernel:
\begin{equation*}
\begin{split}G=\exp{-\frac{x^2+y^2}{2\sigma^2}}\end{split}
\end{equation*}
\sphinxAtStartPar
The Gauss function is a continuous function that extends to infinity. Therefore, we have to define the size of the discrete kernel. A good choise is \(N=2\cdot\lceil 2 \sigma \rceil+1\), multiple of \(\sigma\) can also be set to 2.5 or even 3 but that is almost too much because the boundary weights are very small compared to the central value.

\begin{figure}[htbp]
\centering
\capstart

\noindent\sphinxincludegraphics[scale=0.5]{{gauss_bell}.pdf}
\caption{The shape of a Gaussian filter kernel.}\label{\detokenize{02-ImageEnhancement:id8}}\end{figure}




\begin{equation*}
\begin{split}G=\exp{-\frac{x^2+y^2}{2\,\sigma^2}}\end{split}
\end{equation*}




\sphinxAtStartPar
Example:
\$\(B=\frac{1}{25}\cdot\begin{array}{|c|c|c|c|c|}
\hline
1 & 1 & 1 & 1& 1\\
\hline
1 & 1 & 1 & 1& 1\\
\hline
1 & 1 & 1 & 1& 1\\
\hline
1 & 1 & 1 & 1& 1\\
\hline
1 & 1 & 1 & 1& 1\\
\hline
\end{array}
\)\$



\sphinxAtStartPar
Example:






\section{Different SNR using a Gauss filter}
\label{\detokenize{02-ImageEnhancement:different-snr-using-a-gauss-filter}}
\sphinxAtStartPar
The main purpose of lowpass filters is to reduce the noise in the images. In the following example you can see images with different SNR and what happens when you apply Gauss filters with different \(\sigma\).

\begin{sphinxVerbatim}[commandchars=\\\{\}]
\PYG{n}{fig}\PYG{p}{,} \PYG{n}{ax} \PYG{o}{=} \PYG{n}{plt}\PYG{o}{.}\PYG{n}{subplots}\PYG{p}{(}\PYG{l+m+mi}{3}\PYG{p}{,}\PYG{l+m+mi}{4}\PYG{p}{,}\PYG{n}{figsize}\PYG{o}{=}\PYG{p}{(}\PYG{l+m+mi}{15}\PYG{p}{,}\PYG{l+m+mi}{10}\PYG{p}{)}\PYG{p}{)}\PYG{p}{;} \PYG{n}{ax}\PYG{o}{=}\PYG{n}{ax}\PYG{o}{.}\PYG{n}{ravel}\PYG{p}{(}\PYG{p}{)}
\PYG{n}{img}   \PYG{o}{=} \PYG{n}{plt}\PYG{o}{.}\PYG{n}{imread}\PYG{p}{(}\PYG{l+s+s1}{\PYGZsq{}}\PYG{l+s+s1}{figures/input\PYGZus{}orig.png}\PYG{l+s+s1}{\PYGZsq{}}\PYG{p}{)}\PYG{p}{;} 
\PYG{n}{noise} \PYG{o}{=} \PYG{n}{np}\PYG{o}{.}\PYG{n}{random}\PYG{o}{.}\PYG{n}{normal}\PYG{p}{(}\PYG{l+m+mi}{0}\PYG{p}{,}\PYG{l+m+mi}{1}\PYG{p}{,}\PYG{n}{size}\PYG{o}{=}\PYG{n}{img}\PYG{o}{.}\PYG{n}{shape}\PYG{p}{)}\PYG{p}{;} 
\PYG{n}{SNRs}   \PYG{o}{=} \PYG{p}{[}\PYG{l+m+mi}{1000}\PYG{p}{,}\PYG{l+m+mi}{10}\PYG{p}{,} \PYG{l+m+mi}{5}\PYG{p}{,} \PYG{l+m+mi}{2}\PYG{p}{]}
\PYG{n}{sigmas} \PYG{o}{=} \PYG{p}{[}\PYG{l+m+mf}{0.001}\PYG{p}{,}\PYG{l+m+mi}{1}\PYG{p}{,}\PYG{l+m+mi}{3}\PYG{p}{]}
\PYG{k}{for} \PYG{n}{r}\PYG{p}{,}\PYG{n}{sigma} \PYG{o+ow}{in} \PYG{n+nb}{enumerate}\PYG{p}{(}\PYG{n}{sigmas}\PYG{p}{)} \PYG{p}{:}
    \PYG{k}{for} \PYG{n}{c}\PYG{p}{,}\PYG{n}{SNR} \PYG{o+ow}{in} \PYG{n+nb}{enumerate}\PYG{p}{(}\PYG{n}{SNRs}\PYG{p}{)} \PYG{p}{:}
        \PYG{n}{ax}\PYG{p}{[}\PYG{n}{r}\PYG{o}{*}\PYG{p}{(}\PYG{n+nb}{len}\PYG{p}{(}\PYG{n}{SNRs}\PYG{p}{)}\PYG{p}{)}\PYG{o}{+}\PYG{n}{c}\PYG{p}{]}\PYG{o}{.}\PYG{n}{imshow}\PYG{p}{(}\PYG{n}{ski}\PYG{o}{.}\PYG{n}{filters}\PYG{o}{.}\PYG{n}{gaussian}\PYG{p}{(}\PYG{n}{img}\PYG{o}{+}\PYG{n}{noise}\PYG{o}{/}\PYG{n}{SNR}\PYG{p}{,}\PYG{n}{sigma}\PYG{o}{=}\PYG{n}{sigma}\PYG{p}{)}\PYG{p}{,}\PYG{n}{cmap}\PYG{o}{=}\PYG{l+s+s1}{\PYGZsq{}}\PYG{l+s+s1}{gray}\PYG{l+s+s1}{\PYGZsq{}}\PYG{p}{)}\PYG{p}{;}
    \PYG{n}{ax}\PYG{p}{[}\PYG{n}{r}\PYG{o}{*}\PYG{p}{(}\PYG{n+nb}{len}\PYG{p}{(}\PYG{n}{SNRs}\PYG{p}{)}\PYG{p}{)}\PYG{p}{]}\PYG{o}{.}\PYG{n}{set\PYGZus{}ylabel}\PYG{p}{(}\PYG{l+s+s1}{\PYGZsq{}}\PYG{l+s+s1}{\PYGZdl{}}\PYG{l+s+s1}{\PYGZbs{}}\PYG{l+s+s1}{sigma\PYGZdl{}=}\PYG{l+s+si}{\PYGZob{}0\PYGZcb{}}\PYG{l+s+s1}{\PYGZsq{}}\PYG{o}{.}\PYG{n}{format}\PYG{p}{(}\PYG{n}{sigma}\PYG{p}{)}\PYG{p}{)}
\PYG{k}{for} \PYG{n}{c}\PYG{p}{,}\PYG{n}{SNR} \PYG{o+ow}{in} \PYG{n+nb}{enumerate}\PYG{p}{(}\PYG{n}{SNRs}\PYG{p}{)} \PYG{p}{:}
    \PYG{n}{ax}\PYG{p}{[}\PYG{n}{c}\PYG{p}{]}\PYG{o}{.}\PYG{n}{set\PYGZus{}title}\PYG{p}{(}\PYG{l+s+s1}{\PYGZsq{}}\PYG{l+s+s1}{SNR=}\PYG{l+s+si}{\PYGZob{}0\PYGZcb{}}\PYG{l+s+s1}{\PYGZsq{}}\PYG{o}{.}\PYG{n}{format}\PYG{p}{(}\PYG{n}{SNR}\PYG{p}{)}\PYG{p}{)}
\end{sphinxVerbatim}

\noindent\sphinxincludegraphics{{02-ImageEnhancement_72_0}.png}

\sphinxAtStartPar
What you see in the example is that you will need a wider filter kernel to reduce noise in images with low SNR. The cost of the SNR improvement is unfortunately that fine details in the image are also blurred by the operation. From this example we can conclude that linear filters can’t be applied with out consequences for the image content. Therefore, we have to carefully select filter kernel balancing the improvement in SNR against loss of image features.


\section{How is the convolution computed}
\label{\detokenize{02-ImageEnhancement:how-is-the-convolution-computed}}
\sphinxAtStartPar
Before, we saw that the convolution was comouted using an integral. This is however the definition for continuous variables. In image processing, we change the integral into a sum instead. The convolution is then a weighted sum of the neighborhood pixels. The example below shows how a pixel is updated using a box kernel with the size 3x3. This operation is repeated for all pixels or voxels in the image.

\begin{figure}[htbp]
\centering
\capstart

\noindent\sphinxincludegraphics[scale=0.75]{{principle_mean_filter}.pdf}
\caption{Updating one pixel using a 3x3 box filter.}\label{\detokenize{02-ImageEnhancement:id9}}\end{figure}






\section{Euclidean separability}
\label{\detokenize{02-ImageEnhancement:euclidean-separability}}
\sphinxAtStartPar
The convolution involves quite many additions and multiplications to perform. This can be a bottleneck in a complicated processing workflow. Fortunately, some tricks can be used to recude the number of operations needed for the convolution.

\sphinxAtStartPar
The asociative and commutative laws apply to convoution
\begin{equation*}
\begin{split}(a * b)*c=a*(b*c) \quad \mbox{ and } \quad a * b = b * a \end{split}
\end{equation*}
\sphinxAtStartPar
A convolution kernel is called \sphinxstyleemphasis{separable} if it can be split in two or more parts.

\sphinxAtStartPar
This is the case for the two filter kernels we have seen until now:

\sphinxAtStartPar
\sphinxstylestrong{Box}
\begin{equation*}
\begin{split}\begin{array}{|c|c|c|}
\hline\cdot&\cdot&\cdot\\
\hline\cdot&\cdot&\cdot\\
\hline\cdot&\cdot&\cdot\\
\hline\end{array}=\begin{array}{|c|}\hline\cdot \\
\hline\cdot\\
\hline\cdot\\
\hline\end{array}*
\begin{array}{|c|c|c|}\hline\cdot&\cdot&\cdot\\\hline\end{array}\end{split}
\end{equation*}
\sphinxAtStartPar
\sphinxstylestrong{Gauss}
\begin{equation*}
\begin{split}\exp{-\frac{x^2+y^2}{2\sigma^2}}=\exp{-\frac{x^2}{2\sigma^2}}*\exp{-\frac{y^2}{2\sigma^2}}\end{split}
\end{equation*}

\subsection{Gain of using separable kernels}
\label{\detokenize{02-ImageEnhancement:gain-of-using-separable-kernels}}
\sphinxAtStartPar
Let’s see what it brings to split the kernels in to the principal directions of the image.

\sphinxAtStartPar
Separability reduces the number of computations \(\rightarrow\) faster processing
\begin{itemize}
\item {} 
\sphinxAtStartPar
3\(\times\)3 \(\rightarrow\) 9 mult and 8 add \(\Leftrightarrow\) 6 mult and 4 add

\item {} 
\sphinxAtStartPar
3\(\times\)3\(\times\)3 \(\rightarrow\) 27 mult and 26 add \(\Leftrightarrow\) 9 mult and 6 add

\end{itemize}

\sphinxAtStartPar
The gain is moderate in the 3x3 case, but if we increase the kernel size to 5 instead we see that the gain is increasing radically:
\begin{itemize}
\item {} 
\sphinxAtStartPar
5\(\times\)5 \(\rightarrow\) 25 mult and 24 add \(\Leftrightarrow\) 10 mult and 8 add

\item {} 
\sphinxAtStartPar
5\(\times\)5\(\times\)5 \(\rightarrow\) 125 mult and 124 add \(\Leftrightarrow\) 15 mult and 12 add

\end{itemize}

\sphinxAtStartPar
This looks very promising and it would be great. There are however some things to consider:
\begin{itemize}
\item {} 
\sphinxAtStartPar
Overhead to call the filter function may consume some of the time gain from the separability.

\item {} 
\sphinxAtStartPar
The result may deviate a little depending on the used numerical precision used.

\end{itemize}


\section{The median filter}
\label{\detokenize{02-ImageEnhancement:the-median-filter}}
\sphinxAtStartPar
The median filter is a non\sphinxhyphen{}linear filter with low\sphinxhyphen{}pass characteristics. This filter computes the local median using the pixels in the neighborhood and uses this value in the filtered image.

\begin{figure}[htbp]
\centering
\capstart

\noindent\sphinxincludegraphics[scale=0.75]{{principle_median_filter}.pdf}
\caption{Updating a pixel from its neighborhood using a 3x3 median filter.}\label{\detokenize{02-ImageEnhancement:id10}}\end{figure}




\section{Comparing filters for different noise types}
\label{\detokenize{02-ImageEnhancement:comparing-filters-for-different-noise-types}}
\sphinxAtStartPar
Both linear low\sphinxhyphen{}pass filters and the median filter have SNR improving charactersitics. They do, however, differ in which noise types they are suited for. In the example below you see an image with add Gaussian noise and salt’n’pepper noise.

\begin{sphinxVerbatim}[commandchars=\\\{\}]
\PYG{n}{img} \PYG{o}{=} \PYG{n}{plt}\PYG{o}{.}\PYG{n}{imread}\PYG{p}{(}\PYG{l+s+s1}{\PYGZsq{}}\PYG{l+s+s1}{figures/grasshopper.png}\PYG{l+s+s1}{\PYGZsq{}}\PYG{p}{)}\PYG{p}{;} \PYG{n}{noise}\PYG{o}{=}\PYG{n}{img}\PYG{o}{+}\PYG{n}{np}\PYG{o}{.}\PYG{n}{random}\PYG{o}{.}\PYG{n}{normal}\PYG{p}{(}\PYG{l+m+mi}{0}\PYG{p}{,}\PYG{l+m+mf}{0.1}\PYG{p}{,}\PYG{n}{size}\PYG{o}{=}\PYG{n}{img}\PYG{o}{.}\PYG{n}{shape}\PYG{p}{)}\PYG{p}{;} \PYG{n}{spots}\PYG{o}{=}\PYG{n}{img}\PYG{o}{+}\PYG{l+m+mf}{0.2}\PYG{o}{*}\PYG{n}{snp}\PYG{p}{(}\PYG{n}{img}\PYG{o}{.}\PYG{n}{shape}\PYG{p}{,}\PYG{l+m+mi}{0}\PYG{p}{,}\PYG{l+m+mf}{0.8}\PYG{p}{)}\PYG{p}{;} \PYG{n}{noise}\PYG{o}{=}\PYG{p}{(}\PYG{n}{noise}\PYG{o}{\PYGZhy{}}\PYG{n}{noise}\PYG{o}{.}\PYG{n}{min}\PYG{p}{(}\PYG{p}{)}\PYG{p}{)}\PYG{o}{/}\PYG{p}{(}\PYG{n}{noise}\PYG{o}{.}\PYG{n}{max}\PYG{p}{(}\PYG{p}{)}\PYG{o}{\PYGZhy{}}\PYG{n}{noise}\PYG{o}{.}\PYG{n}{min}\PYG{p}{(}\PYG{p}{)}\PYG{p}{)}\PYG{p}{;}\PYG{n}{spots}\PYG{o}{=}\PYG{p}{(}\PYG{n}{spots}\PYG{o}{\PYGZhy{}}\PYG{n}{spots}\PYG{o}{.}\PYG{n}{min}\PYG{p}{(}\PYG{p}{)}\PYG{p}{)}\PYG{o}{/}\PYG{p}{(}\PYG{n}{spots}\PYG{o}{.}\PYG{n}{max}\PYG{p}{(}\PYG{p}{)}\PYG{o}{\PYGZhy{}}\PYG{n}{spots}\PYG{o}{.}\PYG{n}{min}\PYG{p}{(}\PYG{p}{)}\PYG{p}{)}\PYG{p}{;}
\PYG{n}{plt}\PYG{o}{.}\PYG{n}{figure}\PYG{p}{(}\PYG{n}{figsize}\PYG{o}{=}\PYG{p}{[}\PYG{l+m+mi}{15}\PYG{p}{,}\PYG{l+m+mi}{10}\PYG{p}{]}\PYG{p}{)}\PYG{p}{;} \PYG{n}{vmin}\PYG{o}{=}\PYG{l+m+mf}{0.0}\PYG{p}{;} \PYG{n}{vmax}\PYG{o}{=}\PYG{l+m+mf}{0.9}\PYG{p}{;}
\PYG{n}{plt}\PYG{o}{.}\PYG{n}{subplot}\PYG{p}{(}\PYG{l+m+mi}{2}\PYG{p}{,}\PYG{l+m+mi}{3}\PYG{p}{,}\PYG{l+m+mi}{1}\PYG{p}{)}\PYG{p}{;} \PYG{n}{plt}\PYG{o}{.}\PYG{n}{imshow}\PYG{p}{(}\PYG{n}{spots}\PYG{p}{,}\PYG{n}{vmin}\PYG{o}{=}\PYG{n}{vmin}\PYG{p}{,}\PYG{n}{vmax}\PYG{o}{=}\PYG{n}{vmax}\PYG{p}{)}\PYG{p}{;} \PYG{n}{plt}\PYG{o}{.}\PYG{n}{title}\PYG{p}{(}\PYG{l+s+s1}{\PYGZsq{}}\PYG{l+s+s1}{Spots}\PYG{l+s+s1}{\PYGZsq{}}\PYG{p}{)}\PYG{p}{;} \PYG{n}{plt}\PYG{o}{.}\PYG{n}{axis}\PYG{p}{(}\PYG{l+s+s1}{\PYGZsq{}}\PYG{l+s+s1}{off}\PYG{l+s+s1}{\PYGZsq{}}\PYG{p}{)}\PYG{p}{;}
\PYG{n}{plt}\PYG{o}{.}\PYG{n}{subplot}\PYG{p}{(}\PYG{l+m+mi}{2}\PYG{p}{,}\PYG{l+m+mi}{3}\PYG{p}{,}\PYG{l+m+mi}{2}\PYG{p}{)}\PYG{p}{;} \PYG{n}{plt}\PYG{o}{.}\PYG{n}{imshow}\PYG{p}{(}\PYG{n}{ski}\PYG{o}{.}\PYG{n}{filters}\PYG{o}{.}\PYG{n}{gaussian}\PYG{p}{(}\PYG{n}{spots}\PYG{p}{,}\PYG{n}{sigma}\PYG{o}{=}\PYG{l+m+mi}{1}\PYG{p}{)}\PYG{p}{,}\PYG{n}{vmin}\PYG{o}{=}\PYG{n}{vmin}\PYG{p}{,}\PYG{n}{vmax}\PYG{o}{=}\PYG{n}{vmax}\PYG{p}{)}\PYG{p}{;} \PYG{n}{plt}\PYG{o}{.}\PYG{n}{title}\PYG{p}{(}\PYG{l+s+s1}{\PYGZsq{}}\PYG{l+s+s1}{Gauss filter}\PYG{l+s+s1}{\PYGZsq{}}\PYG{p}{)}\PYG{p}{;} \PYG{n}{plt}\PYG{o}{.}\PYG{n}{axis}\PYG{p}{(}\PYG{l+s+s1}{\PYGZsq{}}\PYG{l+s+s1}{off}\PYG{l+s+s1}{\PYGZsq{}}\PYG{p}{)}\PYG{p}{;}
\PYG{n}{plt}\PYG{o}{.}\PYG{n}{subplot}\PYG{p}{(}\PYG{l+m+mi}{2}\PYG{p}{,}\PYG{l+m+mi}{3}\PYG{p}{,}\PYG{l+m+mi}{3}\PYG{p}{)}\PYG{p}{;} \PYG{n}{plt}\PYG{o}{.}\PYG{n}{imshow}\PYG{p}{(}\PYG{n}{ski}\PYG{o}{.}\PYG{n}{filters}\PYG{o}{.}\PYG{n}{median}\PYG{p}{(}\PYG{n}{spots}\PYG{p}{,}\PYG{n}{disk}\PYG{p}{(}\PYG{l+m+mi}{3}\PYG{p}{)}\PYG{p}{)}\PYG{p}{,}\PYG{n}{vmin}\PYG{o}{=}\PYG{n}{vmin}\PYG{p}{,}\PYG{n}{vmax}\PYG{o}{=}\PYG{n}{vmax}\PYG{p}{)}\PYG{p}{;} \PYG{n}{plt}\PYG{o}{.}\PYG{n}{title}\PYG{p}{(}\PYG{l+s+s1}{\PYGZsq{}}\PYG{l+s+s1}{Median}\PYG{l+s+s1}{\PYGZsq{}}\PYG{p}{)}\PYG{p}{;} \PYG{n}{plt}\PYG{o}{.}\PYG{n}{axis}\PYG{p}{(}\PYG{l+s+s1}{\PYGZsq{}}\PYG{l+s+s1}{off}\PYG{l+s+s1}{\PYGZsq{}}\PYG{p}{)}\PYG{p}{;}

\PYG{n}{plt}\PYG{o}{.}\PYG{n}{subplot}\PYG{p}{(}\PYG{l+m+mi}{2}\PYG{p}{,}\PYG{l+m+mi}{3}\PYG{p}{,}\PYG{l+m+mi}{4}\PYG{p}{)}\PYG{p}{;} \PYG{n}{plt}\PYG{o}{.}\PYG{n}{imshow}\PYG{p}{(}\PYG{n}{noise}\PYG{p}{,}\PYG{n}{vmin}\PYG{o}{=}\PYG{n}{vmin}\PYG{p}{,}\PYG{n}{vmax}\PYG{o}{=}\PYG{n}{vmax}\PYG{p}{)}\PYG{p}{;} \PYG{n}{plt}\PYG{o}{.}\PYG{n}{title}\PYG{p}{(}\PYG{l+s+s1}{\PYGZsq{}}\PYG{l+s+s1}{Gaussian noise}\PYG{l+s+s1}{\PYGZsq{}}\PYG{p}{)}\PYG{p}{;} \PYG{n}{plt}\PYG{o}{.}\PYG{n}{axis}\PYG{p}{(}\PYG{l+s+s1}{\PYGZsq{}}\PYG{l+s+s1}{off}\PYG{l+s+s1}{\PYGZsq{}}\PYG{p}{)}\PYG{p}{;}
\PYG{n}{plt}\PYG{o}{.}\PYG{n}{subplot}\PYG{p}{(}\PYG{l+m+mi}{2}\PYG{p}{,}\PYG{l+m+mi}{3}\PYG{p}{,}\PYG{l+m+mi}{5}\PYG{p}{)}\PYG{p}{;} \PYG{n}{plt}\PYG{o}{.}\PYG{n}{imshow}\PYG{p}{(}\PYG{n}{ski}\PYG{o}{.}\PYG{n}{filters}\PYG{o}{.}\PYG{n}{gaussian}\PYG{p}{(}\PYG{n}{noise}\PYG{p}{,}\PYG{n}{sigma}\PYG{o}{=}\PYG{l+m+mi}{1}\PYG{p}{)}\PYG{p}{,}\PYG{n}{vmin}\PYG{o}{=}\PYG{n}{vmin}\PYG{p}{,}\PYG{n}{vmax}\PYG{o}{=}\PYG{n}{vmax}\PYG{p}{)}\PYG{p}{;} \PYG{n}{plt}\PYG{o}{.}\PYG{n}{title}\PYG{p}{(}\PYG{l+s+s1}{\PYGZsq{}}\PYG{l+s+s1}{Gauss filter}\PYG{l+s+s1}{\PYGZsq{}}\PYG{p}{)}\PYG{p}{;} \PYG{n}{plt}\PYG{o}{.}\PYG{n}{axis}\PYG{p}{(}\PYG{l+s+s1}{\PYGZsq{}}\PYG{l+s+s1}{off}\PYG{l+s+s1}{\PYGZsq{}}\PYG{p}{)}\PYG{p}{;}
\PYG{n}{plt}\PYG{o}{.}\PYG{n}{subplot}\PYG{p}{(}\PYG{l+m+mi}{2}\PYG{p}{,}\PYG{l+m+mi}{3}\PYG{p}{,}\PYG{l+m+mi}{6}\PYG{p}{)}\PYG{p}{;} \PYG{n}{plt}\PYG{o}{.}\PYG{n}{imshow}\PYG{p}{(}\PYG{n}{ski}\PYG{o}{.}\PYG{n}{filters}\PYG{o}{.}\PYG{n}{median}\PYG{p}{(}\PYG{n}{noise}\PYG{p}{,}\PYG{n}{disk}\PYG{p}{(}\PYG{l+m+mi}{3}\PYG{p}{)}\PYG{p}{)}\PYG{p}{,}\PYG{n}{vmin}\PYG{o}{=}\PYG{n}{vmin}\PYG{p}{,}\PYG{n}{vmax}\PYG{o}{=}\PYG{n}{vmax}\PYG{p}{)}\PYG{p}{;} \PYG{n}{plt}\PYG{o}{.}\PYG{n}{title}\PYG{p}{(}\PYG{l+s+s1}{\PYGZsq{}}\PYG{l+s+s1}{Median}\PYG{l+s+s1}{\PYGZsq{}}\PYG{p}{)}\PYG{p}{;} \PYG{n}{plt}\PYG{o}{.}\PYG{n}{axis}\PYG{p}{(}\PYG{l+s+s1}{\PYGZsq{}}\PYG{l+s+s1}{off}\PYG{l+s+s1}{\PYGZsq{}}\PYG{p}{)}\PYG{p}{;}
\end{sphinxVerbatim}

\noindent\sphinxincludegraphics{{02-ImageEnhancement_90_0}.png}

\sphinxAtStartPar
In this comparison, you can clearly see that the median filter is supperior when it comes to remove outliers in an image. In this case, the Gauss filter is only smearing out the outliers, but they still appear as noisy.

\sphinxAtStartPar
In the case of Gaussian noise, it is harder to tell which filter to use. Many are using the median filter as their main goto choise to filter images. Because it is usually gentler to edges and is also good at removing spots. Still, you should be aware of the additional processing time and also that the statistical distribution of the data is not following the original model anymore. The distribution of the data may not be of any concern in many cases, but if you are trying to process the image further based on its distribution, this may result in less good results.


\section{Filter example: Spot cleaning}
\label{\detokenize{02-ImageEnhancement:filter-example-spot-cleaning}}
\begin{figure}[htbp]
\centering
\capstart

\noindent\sphinxincludegraphics[scale=0.5]{{spotty_knot_closeup}.png}
\caption{A neutron image with outliers that we want to remove.}\label{\detokenize{02-ImageEnhancement:id11}}\end{figure}


\begin{itemize}
\item {} 
\sphinxAtStartPar
Many neutron images are corrupted by spots that confuse following processing steps.

\item {} 
\sphinxAtStartPar
The amount, size, and intensity varies with many factors.

\end{itemize}


\begin{itemize}
\item {} 
\sphinxAtStartPar
Low pass filter

\item {} 
\sphinxAtStartPar
Median filter

\item {} 
\sphinxAtStartPar
Detect spots and replace by estimate

\end{itemize}




\section{Spot cleaning algorithm}
\label{\detokenize{02-ImageEnhancement:spot-cleaning-algorithm}}
\begin{figure}[htbp]
\centering
\capstart

\noindent\sphinxincludegraphics[scale=1.0]{{spotclean_algorithm}.pdf}
\caption{Process workflow to selectively remove spots from an image.}\label{\detokenize{02-ImageEnhancement:id12}}\end{figure}



\sphinxAtStartPar
\sphinxstylestrong{Parameters}
\begin{itemize}
\item {} 
\sphinxAtStartPar
\(N\) Width of median filter.

\item {} 
\sphinxAtStartPar
\(k\) Threshold level for outlier detection.

\end{itemize}


\section{Spot cleaning \sphinxhyphen{} Compare performance}
\label{\detokenize{02-ImageEnhancement:spot-cleaning-compare-performance}}
\begin{figure}[htbp]
\centering
\capstart

\noindent\sphinxincludegraphics[scale=1.0]{{spotclean_compare}.pdf}
\caption{Comparing different filters to remove spots from an image.}\label{\detokenize{02-ImageEnhancement:id13}}\end{figure}




\subsection{The ImageJ ways for outlier removal}
\label{\detokenize{02-ImageEnhancement:the-imagej-ways-for-outlier-removal}}
\sphinxAtStartPar
ImageJ is a popular application for interactive image analysis. It offers two ways to remove outliers in the noise menu:
\begin{itemize}
\item {} 
\sphinxAtStartPar
\sphinxstylestrong{Despeckle} Median … please avoid this one!!!

\item {} 
\sphinxAtStartPar
\sphinxstylestrong{Remove outliers} Similar to cleaning described algorithm

\end{itemize}


\section{High\sphinxhyphen{}pass filters}
\label{\detokenize{02-ImageEnhancement:id1}}
\sphinxAtStartPar
High\sphinxhyphen{}pass filters enhance rapid changes \(\rightarrow\) ideal for edge detection


\subsection{Typical high\sphinxhyphen{}pass filters:}
\label{\detokenize{02-ImageEnhancement:typical-high-pass-filters}}

\subsubsection{Gradients}
\label{\detokenize{02-ImageEnhancement:gradients}}\begin{equation*}
\begin{split}\frac{\partial}{\partial x}=\frac{1}{2}\cdot\begin{array}{|c|c|}
\hline
-1 & 1\\
\hline
\end{array}\qquad
\frac{\partial}{\partial x}=\frac{1}{32}\cdot\begin{array}{|c|c|c|}
\hline
-3 & 0 & 3\\
\hline
-10 & 0 & 10\\
\hline
-3 & 0 & 3\\
\hline
\end{array}
\end{split}
\end{equation*}

\subsubsection{Laplacian}
\label{\detokenize{02-ImageEnhancement:laplacian}}\begin{equation*}
\begin{split}
\bigtriangleup=\frac{1}{2}\cdot\begin{array}{|c|c|c|}
\hline
1 & 2 & 1\\
\hline
2 & -12 & 2\\
\hline
1 & 2 & 1\\
\hline
\end{array}
\end{split}
\end{equation*}

\subsubsection{Sobel}
\label{\detokenize{02-ImageEnhancement:sobel}}\begin{equation*}
\begin{split}
G=|\nabla f|=\sqrt{\left(\frac{\partial}{\partial x}f\right)^2 + \left(\frac{\partial}{\partial y}f\right)^2}
\end{split}
\end{equation*}

\section{Gradient example}
\label{\detokenize{02-ImageEnhancement:gradient-example}}

\subsection{Vertical edges}
\label{\detokenize{02-ImageEnhancement:vertical-edges}}
\sphinxAtStartPar
\(\frac{\partial}{\partial x}=\frac{1}{32}\cdot\begin{array}{|c|c|c|}
\hline
-3 & 0 & 3\\
\hline
-10 & 0 & 10\\
\hline
-3 & 0 & 3\\
\hline
\end{array}\)


\subsection{Horizontal egdges}
\label{\detokenize{02-ImageEnhancement:horizontal-egdges}}
\sphinxAtStartPar
\(\frac{\partial}{\partial\,y}=\frac{1}{32}\cdot\begin{array}{|c|c|c|}
\hline
-3 & -10 & -3\\
\hline
0 & 0 & 0\\
\hline
3 & 10 & 3\\
\hline
\end{array}\)

\sphinxAtStartPar
\sphinxhref{https://doi.org/10.1007/3-540-27563-0}{Jaehne, 2005}

\begin{sphinxVerbatim}[commandchars=\\\{\}]
\PYG{n}{img}\PYG{o}{=}\PYG{n}{plt}\PYG{o}{.}\PYG{n}{imread}\PYG{p}{(}\PYG{l+s+s1}{\PYGZsq{}}\PYG{l+s+s1}{figures/orig.png}\PYG{l+s+s1}{\PYGZsq{}}\PYG{p}{)}
\PYG{n}{k} \PYG{o}{=} \PYG{n}{np}\PYG{o}{.}\PYG{n}{array}\PYG{p}{(}\PYG{p}{[}\PYG{p}{[}\PYG{o}{\PYGZhy{}}\PYG{l+m+mi}{3}\PYG{p}{,}\PYG{o}{\PYGZhy{}}\PYG{l+m+mi}{10}\PYG{p}{,}\PYG{o}{\PYGZhy{}}\PYG{l+m+mi}{3}\PYG{p}{]}\PYG{p}{,}\PYG{p}{[}\PYG{l+m+mi}{0}\PYG{p}{,}\PYG{l+m+mi}{0}\PYG{p}{,}\PYG{l+m+mi}{0}\PYG{p}{]}\PYG{p}{,}\PYG{p}{[}\PYG{l+m+mi}{3}\PYG{p}{,}\PYG{l+m+mi}{10}\PYG{p}{,}\PYG{l+m+mi}{3}\PYG{p}{]}\PYG{p}{]}\PYG{p}{)}\PYG{p}{;}
\PYG{n}{plt}\PYG{o}{.}\PYG{n}{figure}\PYG{p}{(}\PYG{n}{figsize}\PYG{o}{=}\PYG{p}{[}\PYG{l+m+mi}{15}\PYG{p}{,}\PYG{l+m+mi}{8}\PYG{p}{]}\PYG{p}{)}
\PYG{n}{plt}\PYG{o}{.}\PYG{n}{subplot}\PYG{p}{(}\PYG{l+m+mi}{1}\PYG{p}{,}\PYG{l+m+mi}{3}\PYG{p}{,}\PYG{l+m+mi}{1}\PYG{p}{)}\PYG{p}{;} \PYG{n}{plt}\PYG{o}{.}\PYG{n}{imshow}\PYG{p}{(}\PYG{n}{img}\PYG{p}{)}\PYG{p}{;}\PYG{n}{plt}\PYG{o}{.}\PYG{n}{title}\PYG{p}{(}\PYG{l+s+s1}{\PYGZsq{}}\PYG{l+s+s1}{Original}\PYG{l+s+s1}{\PYGZsq{}}\PYG{p}{)}\PYG{p}{;}
\PYG{n}{plt}\PYG{o}{.}\PYG{n}{subplot}\PYG{p}{(}\PYG{l+m+mi}{1}\PYG{p}{,}\PYG{l+m+mi}{3}\PYG{p}{,}\PYG{l+m+mi}{2}\PYG{p}{)}\PYG{p}{;} \PYG{n}{plt}\PYG{o}{.}\PYG{n}{imshow}\PYG{p}{(}\PYG{n}{ndimage}\PYG{o}{.}\PYG{n}{convolve}\PYG{p}{(}\PYG{n}{img}\PYG{p}{,}\PYG{n}{np}\PYG{o}{.}\PYG{n}{transpose}\PYG{p}{(}\PYG{n}{k}\PYG{p}{)}\PYG{p}{)}\PYG{p}{)}\PYG{p}{;}\PYG{n}{plt}\PYG{o}{.}\PYG{n}{title}\PYG{p}{(}\PYG{l+s+s1}{\PYGZsq{}}\PYG{l+s+s1}{\PYGZdl{}}\PYG{l+s+s1}{\PYGZbs{}}\PYG{l+s+s1}{partial / }\PYG{l+s+s1}{\PYGZbs{}}\PYG{l+s+s1}{partial x\PYGZdl{}}\PYG{l+s+s1}{\PYGZsq{}}\PYG{p}{)}\PYG{p}{;}
\PYG{n}{plt}\PYG{o}{.}\PYG{n}{subplot}\PYG{p}{(}\PYG{l+m+mi}{1}\PYG{p}{,}\PYG{l+m+mi}{3}\PYG{p}{,}\PYG{l+m+mi}{3}\PYG{p}{)}\PYG{p}{;} \PYG{n}{plt}\PYG{o}{.}\PYG{n}{imshow}\PYG{p}{(}\PYG{n}{ndimage}\PYG{o}{.}\PYG{n}{convolve}\PYG{p}{(}\PYG{n}{img}\PYG{p}{,}\PYG{n}{k}\PYG{p}{)}\PYG{p}{)}\PYG{p}{;}\PYG{n}{plt}\PYG{o}{.}\PYG{n}{title}\PYG{p}{(}\PYG{l+s+s1}{\PYGZsq{}}\PYG{l+s+s1}{\PYGZdl{}}\PYG{l+s+s1}{\PYGZbs{}}\PYG{l+s+s1}{partial / }\PYG{l+s+s1}{\PYGZbs{}}\PYG{l+s+s1}{partial y\PYGZdl{}}\PYG{l+s+s1}{\PYGZsq{}}\PYG{p}{)}\PYG{p}{;}
\end{sphinxVerbatim}

\noindent\sphinxincludegraphics{{02-ImageEnhancement_107_0}.png}


\section{Edge detection examples}
\label{\detokenize{02-ImageEnhancement:edge-detection-examples}}
\begin{sphinxVerbatim}[commandchars=\\\{\}]
\PYG{n}{img}\PYG{o}{=}\PYG{n}{plt}\PYG{o}{.}\PYG{n}{imread}\PYG{p}{(}\PYG{l+s+s1}{\PYGZsq{}}\PYG{l+s+s1}{figures/orig.png}\PYG{l+s+s1}{\PYGZsq{}}\PYG{p}{)}\PYG{p}{;}
\PYG{n}{plt}\PYG{o}{.}\PYG{n}{figure}\PYG{p}{(}\PYG{n}{figsize}\PYG{o}{=}\PYG{p}{[}\PYG{l+m+mi}{15}\PYG{p}{,}\PYG{l+m+mi}{6}\PYG{p}{]}\PYG{p}{)}
\PYG{n}{plt}\PYG{o}{.}\PYG{n}{subplot}\PYG{p}{(}\PYG{l+m+mi}{1}\PYG{p}{,}\PYG{l+m+mi}{2}\PYG{p}{,}\PYG{l+m+mi}{1}\PYG{p}{)}\PYG{p}{;}\PYG{n}{plt}\PYG{o}{.}\PYG{n}{imshow}\PYG{p}{(}\PYG{n}{ski}\PYG{o}{.}\PYG{n}{filters}\PYG{o}{.}\PYG{n}{laplace}\PYG{p}{(}\PYG{n}{img}\PYG{p}{)}\PYG{p}{,}\PYG{n}{clim}\PYG{o}{=}\PYG{p}{[}\PYG{o}{\PYGZhy{}}\PYG{l+m+mf}{0.08}\PYG{p}{,}\PYG{l+m+mf}{0.08}\PYG{p}{]}\PYG{p}{)}\PYG{p}{;} \PYG{n}{plt}\PYG{o}{.}\PYG{n}{title}\PYG{p}{(}\PYG{l+s+s1}{\PYGZsq{}}\PYG{l+s+s1}{Laplacian}\PYG{l+s+s1}{\PYGZsq{}}\PYG{p}{)}\PYG{p}{;} \PYG{n}{plt}\PYG{o}{.}\PYG{n}{colorbar}\PYG{p}{(}\PYG{p}{)}\PYG{p}{;}
\PYG{n}{plt}\PYG{o}{.}\PYG{n}{subplot}\PYG{p}{(}\PYG{l+m+mi}{1}\PYG{p}{,}\PYG{l+m+mi}{2}\PYG{p}{,}\PYG{l+m+mi}{2}\PYG{p}{)}\PYG{p}{;}\PYG{n}{plt}\PYG{o}{.}\PYG{n}{imshow}\PYG{p}{(}\PYG{n}{ski}\PYG{o}{.}\PYG{n}{filters}\PYG{o}{.}\PYG{n}{sobel}\PYG{p}{(}\PYG{n}{img}\PYG{p}{)}\PYG{p}{)}\PYG{p}{;} \PYG{n}{plt}\PYG{o}{.}\PYG{n}{title}\PYG{p}{(}\PYG{l+s+s1}{\PYGZsq{}}\PYG{l+s+s1}{Sobel}\PYG{l+s+s1}{\PYGZsq{}}\PYG{p}{)}\PYG{p}{;}  \PYG{n}{plt}\PYG{o}{.}\PYG{n}{colorbar}\PYG{p}{(}\PYG{p}{)}\PYG{p}{;}
\end{sphinxVerbatim}

\noindent\sphinxincludegraphics{{02-ImageEnhancement_109_0}.png}


\section{Relevance of filters to machine learning}
\label{\detokenize{02-ImageEnhancement:relevance-of-filters-to-machine-learning}}
\begin{figure}[htbp]
\centering
\capstart

\noindent\sphinxincludegraphics[scale=0.8]{{convolutional_neural_network}.png}
\caption{An example of a convolutional neural network for image classification.}\label{\detokenize{02-ImageEnhancement:id14}}\end{figure}



\sphinxAtStartPar
\sphinxhref{https://developer.nvidia.com/discover/convolutional-neural-network}{NVIDIA Developer zone}


\chapter{Frequency space filters}
\label{\detokenize{02-ImageEnhancement:frequency-space-filters}}
\sphinxAtStartPar
We already mentioned that the there are frequencies to filter in images. There is a whole collection of filters that operate directly in frequency space. This is mostly done using the Fourier transform and its inverse.


\section{Applications of the Fourier transform}
\label{\detokenize{02-ImageEnhancement:applications-of-the-fourier-transform}}

\section{The Fourier transform}
\label{\detokenize{02-ImageEnhancement:the-fourier-transform}}
\sphinxAtStartPar
You may only know the Fourier transform for the 1D case. It is however very easy to increase the number of dimensions. Below you see the 2D Fourier transform and its inverse. These operations are lossless one\sphinxhyphen{}to\sphinxhyphen{}one, meaning you can go from one domain to the other without loosing any information.

\sphinxAtStartPar
\sphinxstylestrong{Transform}
\begin{equation*}
\begin{split}G(\xi_1,\xi_2)=\mathcal{F}\{g\}=\int_{-\infty}^{\infty}\int_{-\infty}^{\infty} g(x,y)
\exp{-i(\xi_1 x+\xi_2 y)}\,dx\,dy\end{split}
\end{equation*}
\sphinxAtStartPar
\sphinxstylestrong{Inverse}
\begin{equation*}
\begin{split} g(x,y)=\mathcal{F}^{-1}\{G\}=\frac{1}{(2\,\pi)^2}\int_{-\infty}^{\infty}\int_{-\infty}^{\infty} G(\omega)
\exp{i(\xi_1 x+\xi_2 y)} d\xi_1 d\xi_2\end{split}
\end{equation*}

\subsection{FFT (Fast Fourier Transform)}
\label{\detokenize{02-ImageEnhancement:fft-fast-fourier-transform}}
\sphinxAtStartPar
The previous equations are meant for continuous signals. Images are discrete signals and the integrals turn into sums. This may introduce some numerical losses when the transforms are computed. Computing the DFT is an \(O(N^2)\) operation. There is a way to speed this up by using the Fast Fourier Transform algorithm. This algorithm requires that the data has the length \(N=2^k\). In this case the complexity reduces to \(O(N\cdot\log(N))\), which is a radical speed\sphinxhyphen{}up.

\sphinxAtStartPar
In practice \sphinxhyphen{} you never see the transform equations.
The Fast Fourier Transform algorithm is available in numerical libraries and tools.

\sphinxAtStartPar
\sphinxhref{https://doi.org/10.1007/3-540-27563-0}{Jaehne, 2005}


\section{Some mathematical features of the FT}
\label{\detokenize{02-ImageEnhancement:some-mathematical-features-of-the-ft}}

\subsection{Additition}
\label{\detokenize{02-ImageEnhancement:additition}}
\sphinxAtStartPar
Addition of Fourier spectra works the same way as for real space signal. The reason is that the transform is based on summarions. You probably proved this as an exercise in your math classes. The plots below shows the spectra of the sum of two signals.

\sphinxAtStartPar
\(\mathcal{F}\{a+b\} = \mathcal{F}\{a\}+\mathcal{F}\{b\}\)

\begin{sphinxVerbatim}[commandchars=\\\{\}]
\PYG{n}{x} \PYG{o}{=} \PYG{n}{np}\PYG{o}{.}\PYG{n}{linspace}\PYG{p}{(}\PYG{l+m+mi}{0}\PYG{p}{,}\PYG{l+m+mi}{50}\PYG{p}{,}\PYG{l+m+mi}{100}\PYG{p}{)}\PYG{p}{;} \PYG{n}{s0}\PYG{o}{=}\PYG{n}{np}\PYG{o}{.}\PYG{n}{sin}\PYG{p}{(}\PYG{l+m+mf}{0.5}\PYG{o}{*}\PYG{n}{x}\PYG{p}{)}\PYG{p}{;} \PYG{n}{s1}\PYG{o}{=}\PYG{n}{np}\PYG{o}{.}\PYG{n}{sin}\PYG{p}{(}\PYG{l+m+mi}{2}\PYG{o}{*}\PYG{n}{x}\PYG{p}{)}\PYG{p}{;}
\PYG{n}{plt}\PYG{o}{.}\PYG{n}{figure}\PYG{p}{(}\PYG{n}{figsize}\PYG{o}{=}\PYG{p}{[}\PYG{l+m+mi}{15}\PYG{p}{,}\PYG{l+m+mi}{5}\PYG{p}{]}\PYG{p}{)}
\PYG{n}{plt}\PYG{o}{.}\PYG{n}{subplot}\PYG{p}{(}\PYG{l+m+mi}{2}\PYG{p}{,}\PYG{l+m+mi}{3}\PYG{p}{,}\PYG{l+m+mi}{1}\PYG{p}{)}\PYG{p}{;}\PYG{n}{plt}\PYG{o}{.}\PYG{n}{plot}\PYG{p}{(}\PYG{n}{x}\PYG{p}{,}\PYG{n}{s0}\PYG{p}{)}\PYG{p}{;} \PYG{n}{plt}\PYG{o}{.}\PYG{n}{axis}\PYG{p}{(}\PYG{l+s+s1}{\PYGZsq{}}\PYG{l+s+s1}{off}\PYG{l+s+s1}{\PYGZsq{}}\PYG{p}{)}\PYG{p}{;}\PYG{n}{plt}\PYG{o}{.}\PYG{n}{title}\PYG{p}{(}\PYG{l+s+s1}{\PYGZsq{}}\PYG{l+s+s1}{\PYGZdl{}s\PYGZus{}0\PYGZdl{}}\PYG{l+s+s1}{\PYGZsq{}}\PYG{p}{)}\PYG{p}{;}\PYG{n}{plt}\PYG{o}{.}\PYG{n}{subplot}\PYG{p}{(}\PYG{l+m+mi}{2}\PYG{p}{,}\PYG{l+m+mi}{3}\PYG{p}{,}\PYG{l+m+mi}{4}\PYG{p}{)}\PYG{p}{;}\PYG{n}{plt}\PYG{o}{.}\PYG{n}{plot}\PYG{p}{(}\PYG{n}{np}\PYG{o}{.}\PYG{n}{abs}\PYG{p}{(}\PYG{n}{np}\PYG{o}{.}\PYG{n}{fft}\PYG{o}{.}\PYG{n}{fftshift}\PYG{p}{(}\PYG{n}{np}\PYG{o}{.}\PYG{n}{fft}\PYG{o}{.}\PYG{n}{fft}\PYG{p}{(}\PYG{n}{s0}\PYG{p}{)}\PYG{p}{)}\PYG{p}{)}\PYG{p}{)}\PYG{p}{;}\PYG{n}{plt}\PYG{o}{.}\PYG{n}{axis}\PYG{p}{(}\PYG{l+s+s1}{\PYGZsq{}}\PYG{l+s+s1}{off}\PYG{l+s+s1}{\PYGZsq{}}\PYG{p}{)}\PYG{p}{;}
\PYG{n}{plt}\PYG{o}{.}\PYG{n}{subplot}\PYG{p}{(}\PYG{l+m+mi}{2}\PYG{p}{,}\PYG{l+m+mi}{3}\PYG{p}{,}\PYG{l+m+mi}{2}\PYG{p}{)}\PYG{p}{;}\PYG{n}{plt}\PYG{o}{.}\PYG{n}{plot}\PYG{p}{(}\PYG{n}{x}\PYG{p}{,}\PYG{n}{s1}\PYG{p}{)}\PYG{p}{;} \PYG{n}{plt}\PYG{o}{.}\PYG{n}{axis}\PYG{p}{(}\PYG{l+s+s1}{\PYGZsq{}}\PYG{l+s+s1}{off}\PYG{l+s+s1}{\PYGZsq{}}\PYG{p}{)}\PYG{p}{;}\PYG{n}{plt}\PYG{o}{.}\PYG{n}{title}\PYG{p}{(}\PYG{l+s+s1}{\PYGZsq{}}\PYG{l+s+s1}{\PYGZdl{}s\PYGZus{}1\PYGZdl{}}\PYG{l+s+s1}{\PYGZsq{}}\PYG{p}{)}\PYG{p}{;}\PYG{n}{plt}\PYG{o}{.}\PYG{n}{subplot}\PYG{p}{(}\PYG{l+m+mi}{2}\PYG{p}{,}\PYG{l+m+mi}{3}\PYG{p}{,}\PYG{l+m+mi}{5}\PYG{p}{)}\PYG{p}{;}\PYG{n}{plt}\PYG{o}{.}\PYG{n}{plot}\PYG{p}{(}\PYG{n}{np}\PYG{o}{.}\PYG{n}{abs}\PYG{p}{(}\PYG{n}{np}\PYG{o}{.}\PYG{n}{fft}\PYG{o}{.}\PYG{n}{fftshift}\PYG{p}{(}\PYG{n}{np}\PYG{o}{.}\PYG{n}{fft}\PYG{o}{.}\PYG{n}{fft}\PYG{p}{(}\PYG{n}{s1}\PYG{p}{)}\PYG{p}{)}\PYG{p}{)}\PYG{p}{)}\PYG{p}{;}\PYG{n}{plt}\PYG{o}{.}\PYG{n}{axis}\PYG{p}{(}\PYG{l+s+s1}{\PYGZsq{}}\PYG{l+s+s1}{off}\PYG{l+s+s1}{\PYGZsq{}}\PYG{p}{)}\PYG{p}{;}
\PYG{n}{plt}\PYG{o}{.}\PYG{n}{subplot}\PYG{p}{(}\PYG{l+m+mi}{2}\PYG{p}{,}\PYG{l+m+mi}{3}\PYG{p}{,}\PYG{l+m+mi}{3}\PYG{p}{)}\PYG{p}{;}\PYG{n}{plt}\PYG{o}{.}\PYG{n}{plot}\PYG{p}{(}\PYG{n}{x}\PYG{p}{,}\PYG{n}{s0}\PYG{o}{+}\PYG{n}{s1}\PYG{p}{)}\PYG{p}{;} \PYG{n}{plt}\PYG{o}{.}\PYG{n}{axis}\PYG{p}{(}\PYG{l+s+s1}{\PYGZsq{}}\PYG{l+s+s1}{off}\PYG{l+s+s1}{\PYGZsq{}}\PYG{p}{)}\PYG{p}{;}\PYG{n}{plt}\PYG{o}{.}\PYG{n}{title}\PYG{p}{(}\PYG{l+s+s1}{\PYGZsq{}}\PYG{l+s+s1}{\PYGZdl{}s\PYGZus{}0+s\PYGZus{}1\PYGZdl{}}\PYG{l+s+s1}{\PYGZsq{}}\PYG{p}{)}\PYG{p}{;}\PYG{n}{plt}\PYG{o}{.}\PYG{n}{subplot}\PYG{p}{(}\PYG{l+m+mi}{2}\PYG{p}{,}\PYG{l+m+mi}{3}\PYG{p}{,}\PYG{l+m+mi}{6}\PYG{p}{)}\PYG{p}{;}\PYG{n}{plt}\PYG{o}{.}\PYG{n}{plot}\PYG{p}{(}\PYG{n}{np}\PYG{o}{.}\PYG{n}{abs}\PYG{p}{(}\PYG{n}{np}\PYG{o}{.}\PYG{n}{fft}\PYG{o}{.}\PYG{n}{fftshift}\PYG{p}{(}\PYG{n}{np}\PYG{o}{.}\PYG{n}{fft}\PYG{o}{.}\PYG{n}{fft}\PYG{p}{(}\PYG{n}{s0}\PYG{o}{+}\PYG{n}{s1}\PYG{p}{)}\PYG{p}{)}\PYG{p}{)}\PYG{p}{)}\PYG{p}{;}\PYG{n}{plt}\PYG{o}{.}\PYG{n}{axis}\PYG{p}{(}\PYG{l+s+s1}{\PYGZsq{}}\PYG{l+s+s1}{off}\PYG{l+s+s1}{\PYGZsq{}}\PYG{p}{)}\PYG{p}{;}
\end{sphinxVerbatim}

\noindent\sphinxincludegraphics{{02-ImageEnhancement_124_0}.png}


\subsection{Convolution}
\label{\detokenize{02-ImageEnhancement:convolution}}
\sphinxAtStartPar
Convolution, which is a relatively intense task to perform in real space reduces to a frequency\sphinxhyphen{}wise multiplication in the Fourier space. This is a very useful property of the transform. It allows to design some filters much easier than in real space. In particular, band stop filters to remove a specific feature in the image. Also filters with wide kernels can be faster to compute using the Fourier transform.

\sphinxAtStartPar
\(\mathcal{F}\{a * b\} = \mathcal{F}\{a\} \cdot \mathcal{F}\{b\} \)
\(\mathcal{F}\{a \cdot b\} = \mathcal{F}\{a\} * \mathcal{F}\{b\} \)


\section{Additive noise in Fourier space}
\label{\detokenize{02-ImageEnhancement:additive-noise-in-fourier-space}}

\subsection{Real space}
\label{\detokenize{02-ImageEnhancement:real-space}}
\begin{sphinxVerbatim}[commandchars=\\\{\}]
\PYG{n}{img}\PYG{o}{=}\PYG{n}{plt}\PYG{o}{.}\PYG{n}{imread}\PYG{p}{(}\PYG{l+s+s1}{\PYGZsq{}}\PYG{l+s+s1}{figures/bp\PYGZus{}ex\PYGZus{}original.png}\PYG{l+s+s1}{\PYGZsq{}}\PYG{p}{)}\PYG{p}{;} \PYG{n}{noise}\PYG{o}{=}\PYG{n}{np}\PYG{o}{.}\PYG{n}{random}\PYG{o}{.}\PYG{n}{normal}\PYG{p}{(}\PYG{l+m+mi}{0}\PYG{p}{,}\PYG{l+m+mf}{0.2}\PYG{p}{,}\PYG{n}{size}\PYG{o}{=}\PYG{n}{img}\PYG{o}{.}\PYG{n}{shape}\PYG{p}{)}\PYG{p}{;} \PYG{n}{nimg}\PYG{o}{=}\PYG{n}{img}\PYG{o}{+}\PYG{n}{noise}\PYG{p}{;}
\PYG{n}{plt}\PYG{o}{.}\PYG{n}{figure}\PYG{p}{(}\PYG{n}{figsize}\PYG{o}{=}\PYG{p}{[}\PYG{l+m+mi}{15}\PYG{p}{,}\PYG{l+m+mi}{3}\PYG{p}{]}\PYG{p}{)}
\PYG{n}{plt}\PYG{o}{.}\PYG{n}{subplot}\PYG{p}{(}\PYG{l+m+mi}{1}\PYG{p}{,}\PYG{l+m+mi}{3}\PYG{p}{,}\PYG{l+m+mi}{1}\PYG{p}{)}\PYG{p}{;} \PYG{n}{plt}\PYG{o}{.}\PYG{n}{imshow}\PYG{p}{(}\PYG{n}{img}\PYG{p}{)}\PYG{p}{;}\PYG{n}{plt}\PYG{o}{.}\PYG{n}{title}\PYG{p}{(}\PYG{l+s+s1}{\PYGZsq{}}\PYG{l+s+s1}{Image}\PYG{l+s+s1}{\PYGZsq{}}\PYG{p}{)}\PYG{p}{;}
\PYG{n}{plt}\PYG{o}{.}\PYG{n}{subplot}\PYG{p}{(}\PYG{l+m+mi}{1}\PYG{p}{,}\PYG{l+m+mi}{3}\PYG{p}{,}\PYG{l+m+mi}{2}\PYG{p}{)}\PYG{p}{;} \PYG{n}{plt}\PYG{o}{.}\PYG{n}{imshow}\PYG{p}{(}\PYG{n}{noise}\PYG{p}{)}\PYG{p}{;}\PYG{n}{plt}\PYG{o}{.}\PYG{n}{title}\PYG{p}{(}\PYG{l+s+s1}{\PYGZsq{}}\PYG{l+s+s1}{Noise}\PYG{l+s+s1}{\PYGZsq{}}\PYG{p}{)}\PYG{p}{;}
\PYG{n}{plt}\PYG{o}{.}\PYG{n}{subplot}\PYG{p}{(}\PYG{l+m+mi}{1}\PYG{p}{,}\PYG{l+m+mi}{3}\PYG{p}{,}\PYG{l+m+mi}{3}\PYG{p}{)}\PYG{p}{;} \PYG{n}{plt}\PYG{o}{.}\PYG{n}{imshow}\PYG{p}{(}\PYG{n}{nimg}\PYG{p}{)}\PYG{p}{;}\PYG{n}{plt}\PYG{o}{.}\PYG{n}{title}\PYG{p}{(}\PYG{l+s+s1}{\PYGZsq{}}\PYG{l+s+s1}{Image + noise}\PYG{l+s+s1}{\PYGZsq{}}\PYG{p}{)}\PYG{p}{;}
\end{sphinxVerbatim}

\noindent\sphinxincludegraphics{{02-ImageEnhancement_129_0}.png}


\subsection{Fourier space}
\label{\detokenize{02-ImageEnhancement:fourier-space}}
\begin{sphinxVerbatim}[commandchars=\\\{\}]
\PYG{n}{plt}\PYG{o}{.}\PYG{n}{figure}\PYG{p}{(}\PYG{n}{figsize}\PYG{o}{=}\PYG{p}{[}\PYG{l+m+mi}{15}\PYG{p}{,}\PYG{l+m+mi}{3}\PYG{p}{]}\PYG{p}{)}
\PYG{n}{plt}\PYG{o}{.}\PYG{n}{subplot}\PYG{p}{(}\PYG{l+m+mi}{1}\PYG{p}{,}\PYG{l+m+mi}{3}\PYG{p}{,}\PYG{l+m+mi}{1}\PYG{p}{)}\PYG{p}{;} \PYG{n}{plt}\PYG{o}{.}\PYG{n}{imshow}\PYG{p}{(}\PYG{n}{np}\PYG{o}{.}\PYG{n}{log}\PYG{p}{(}\PYG{n}{np}\PYG{o}{.}\PYG{n}{abs}\PYG{p}{(}\PYG{n}{np}\PYG{o}{.}\PYG{n}{fft}\PYG{o}{.}\PYG{n}{fftshift}\PYG{p}{(}\PYG{n}{np}\PYG{o}{.}\PYG{n}{fft}\PYG{o}{.}\PYG{n}{fft2}\PYG{p}{(}\PYG{n}{img}\PYG{p}{)}\PYG{p}{)}\PYG{p}{)}\PYG{p}{)}\PYG{p}{)}\PYG{p}{;}\PYG{n}{plt}\PYG{o}{.}\PYG{n}{title}\PYG{p}{(}\PYG{l+s+s1}{\PYGZsq{}}\PYG{l+s+s1}{Image}\PYG{l+s+s1}{\PYGZsq{}}\PYG{p}{)}\PYG{p}{;}
\PYG{n}{plt}\PYG{o}{.}\PYG{n}{subplot}\PYG{p}{(}\PYG{l+m+mi}{1}\PYG{p}{,}\PYG{l+m+mi}{3}\PYG{p}{,}\PYG{l+m+mi}{2}\PYG{p}{)}\PYG{p}{;} \PYG{n}{plt}\PYG{o}{.}\PYG{n}{imshow}\PYG{p}{(}\PYG{n}{np}\PYG{o}{.}\PYG{n}{log}\PYG{p}{(}\PYG{n}{np}\PYG{o}{.}\PYG{n}{abs}\PYG{p}{(}\PYG{n}{np}\PYG{o}{.}\PYG{n}{fft}\PYG{o}{.}\PYG{n}{fftshift}\PYG{p}{(}\PYG{n}{np}\PYG{o}{.}\PYG{n}{fft}\PYG{o}{.}\PYG{n}{fft2}\PYG{p}{(}\PYG{n}{noise}\PYG{p}{)}\PYG{p}{)}\PYG{p}{)}\PYG{p}{)}\PYG{p}{)}\PYG{p}{;}\PYG{n}{plt}\PYG{o}{.}\PYG{n}{title}\PYG{p}{(}\PYG{l+s+s1}{\PYGZsq{}}\PYG{l+s+s1}{Noise}\PYG{l+s+s1}{\PYGZsq{}}\PYG{p}{)}\PYG{p}{;}
\PYG{n}{plt}\PYG{o}{.}\PYG{n}{subplot}\PYG{p}{(}\PYG{l+m+mi}{1}\PYG{p}{,}\PYG{l+m+mi}{3}\PYG{p}{,}\PYG{l+m+mi}{3}\PYG{p}{)}\PYG{p}{;} \PYG{n}{plt}\PYG{o}{.}\PYG{n}{imshow}\PYG{p}{(}\PYG{n}{np}\PYG{o}{.}\PYG{n}{log}\PYG{p}{(}\PYG{n}{np}\PYG{o}{.}\PYG{n}{abs}\PYG{p}{(}\PYG{n}{np}\PYG{o}{.}\PYG{n}{fft}\PYG{o}{.}\PYG{n}{fftshift}\PYG{p}{(}\PYG{n}{np}\PYG{o}{.}\PYG{n}{fft}\PYG{o}{.}\PYG{n}{fft2}\PYG{p}{(}\PYG{n}{nimg}\PYG{p}{)}\PYG{p}{)}\PYG{p}{)}\PYG{p}{)}\PYG{p}{)}\PYG{p}{;}\PYG{n}{plt}\PYG{o}{.}\PYG{n}{title}\PYG{p}{(}\PYG{l+s+s1}{\PYGZsq{}}\PYG{l+s+s1}{Image + noise}\PYG{l+s+s1}{\PYGZsq{}}\PYG{p}{)}\PYG{p}{;}
\end{sphinxVerbatim}

\noindent\sphinxincludegraphics{{02-ImageEnhancement_131_0}.png}


\subsection{Problem}
\label{\detokenize{02-ImageEnhancement:problem}}
\sphinxAtStartPar
How can we suppress noise without destroying relevant image features?


\section{Spatial frequencies and orientation}
\label{\detokenize{02-ImageEnhancement:spatial-frequencies-and-orientation}}
\begin{sphinxVerbatim}[commandchars=\\\{\}]
\PYG{k}{def} \PYG{n+nf}{ripple}\PYG{p}{(}\PYG{n}{size}\PYG{o}{=}\PYG{l+m+mi}{128}\PYG{p}{,}\PYG{n}{angle}\PYG{o}{=}\PYG{l+m+mi}{0}\PYG{p}{,}\PYG{n}{w0}\PYG{o}{=}\PYG{l+m+mf}{0.1}\PYG{p}{)} \PYG{p}{:}
    \PYG{n}{w}\PYG{o}{=}\PYG{n}{w0}\PYG{o}{*}\PYG{n}{np}\PYG{o}{.}\PYG{n}{linspace}\PYG{p}{(}\PYG{l+m+mi}{0}\PYG{p}{,}\PYG{l+m+mi}{1}\PYG{p}{,}\PYG{n}{size}\PYG{p}{)}\PYG{p}{;}
    \PYG{p}{[}\PYG{n}{x}\PYG{p}{,}\PYG{n}{y}\PYG{p}{]}\PYG{o}{=}\PYG{n}{np}\PYG{o}{.}\PYG{n}{meshgrid}\PYG{p}{(}\PYG{n}{w}\PYG{p}{,}\PYG{n}{w}\PYG{p}{)}\PYG{p}{;}
    \PYG{n}{img}\PYG{o}{=}\PYG{n}{np}\PYG{o}{.}\PYG{n}{sin}\PYG{p}{(}\PYG{p}{(}\PYG{n}{np}\PYG{o}{.}\PYG{n}{sin}\PYG{p}{(}\PYG{n}{angle}\PYG{p}{)}\PYG{o}{*}\PYG{n}{x}\PYG{p}{)}\PYG{o}{+}\PYG{p}{(}\PYG{n}{np}\PYG{o}{.}\PYG{n}{cos}\PYG{p}{(}\PYG{n}{angle}\PYG{p}{)}\PYG{o}{*}\PYG{n}{y}\PYG{p}{)}\PYG{p}{)}\PYG{p}{;}
    \PYG{k}{return} \PYG{n}{img}
\PYG{n}{N}\PYG{o}{=}\PYG{l+m+mi}{64}\PYG{p}{;}
\PYG{n}{fig}\PYG{p}{,} \PYG{n}{ax} \PYG{o}{=} \PYG{n}{plt}\PYG{o}{.}\PYG{n}{subplots}\PYG{p}{(}\PYG{l+m+mi}{2}\PYG{p}{,}\PYG{l+m+mi}{4}\PYG{p}{,} \PYG{n}{figsize}\PYG{o}{=}\PYG{p}{(}\PYG{l+m+mi}{15}\PYG{p}{,}\PYG{l+m+mi}{6}\PYG{p}{)}\PYG{p}{)}\PYG{p}{;} \PYG{n}{ax}\PYG{o}{=}\PYG{n}{ax}\PYG{o}{.}\PYG{n}{ravel}\PYG{p}{(}\PYG{p}{)}

\PYG{k}{for} \PYG{n}{idx}\PYG{p}{,}\PYG{n}{angle} \PYG{o+ow}{in} \PYG{n+nb}{enumerate}\PYG{p}{(}\PYG{p}{[}\PYG{l+m+mi}{0}\PYG{p}{,}\PYG{l+m+mi}{30}\PYG{p}{,}\PYG{l+m+mi}{60}\PYG{p}{,}\PYG{l+m+mi}{90}\PYG{p}{]}\PYG{p}{)} \PYG{p}{:}
    \PYG{n}{d}\PYG{o}{=}\PYG{n}{ripple}\PYG{p}{(}\PYG{n}{N}\PYG{p}{,}\PYG{n}{angle}\PYG{o}{=}\PYG{p}{(}\PYG{n}{angle}\PYG{o}{+}\PYG{l+m+mf}{0.1}\PYG{p}{)}\PYG{o}{/}\PYG{l+m+mi}{180}\PYG{o}{*}\PYG{n}{np}\PYG{o}{.}\PYG{n}{pi}\PYG{p}{,}\PYG{n}{w0}\PYG{o}{=}\PYG{l+m+mi}{100}\PYG{p}{)}\PYG{p}{;}
    \PYG{n}{ax}\PYG{p}{[}\PYG{n}{idx}\PYG{p}{]}\PYG{o}{.}\PYG{n}{imshow}\PYG{p}{(}\PYG{n}{d}\PYG{p}{)}\PYG{p}{;} \PYG{n}{ax}\PYG{p}{[}\PYG{n}{idx}\PYG{p}{]}\PYG{o}{.}\PYG{n}{set\PYGZus{}title}\PYG{p}{(}\PYG{l+s+s1}{\PYGZsq{}}\PYG{l+s+s1}{\PYGZdl{}}\PYG{l+s+si}{\PYGZob{}0\PYGZcb{}}\PYG{l+s+s1}{ degrees\PYGZdl{}}\PYG{l+s+s1}{\PYGZsq{}}\PYG{o}{.}\PYG{n}{format}\PYG{p}{(}\PYG{n}{angle}\PYG{p}{)}\PYG{p}{)}\PYG{p}{;} \PYG{n}{ax}\PYG{p}{[}\PYG{n}{idx}\PYG{p}{]}\PYG{o}{.}\PYG{n}{axis}\PYG{p}{(}\PYG{l+s+s1}{\PYGZsq{}}\PYG{l+s+s1}{off}\PYG{l+s+s1}{\PYGZsq{}}\PYG{p}{)}\PYG{p}{;}
    \PYG{n}{ax}\PYG{p}{[}\PYG{n}{idx}\PYG{o}{+}\PYG{l+m+mi}{4}\PYG{p}{]}\PYG{o}{.}\PYG{n}{imshow}\PYG{p}{(}\PYG{p}{(}\PYG{n}{np}\PYG{o}{.}\PYG{n}{abs}\PYG{p}{(}\PYG{n}{np}\PYG{o}{.}\PYG{n}{fft}\PYG{o}{.}\PYG{n}{fftshift}\PYG{p}{(}\PYG{n}{np}\PYG{o}{.}\PYG{n}{fft}\PYG{o}{.}\PYG{n}{fft2}\PYG{p}{(}\PYG{n}{d}\PYG{p}{)}\PYG{p}{)}\PYG{p}{)}\PYG{p}{)}\PYG{p}{)}\PYG{p}{;}   \PYG{n}{ax}\PYG{p}{[}\PYG{n}{idx}\PYG{o}{+}\PYG{l+m+mi}{4}\PYG{p}{]}\PYG{o}{.}\PYG{n}{axis}\PYG{p}{(}\PYG{l+s+s1}{\PYGZsq{}}\PYG{l+s+s1}{off}\PYG{l+s+s1}{\PYGZsq{}}\PYG{p}{)}\PYG{p}{;}
\end{sphinxVerbatim}

\noindent\sphinxincludegraphics{{02-ImageEnhancement_134_0}.png}


\section{Example \sphinxhyphen{} Stripe removal in Fourier space}
\label{\detokenize{02-ImageEnhancement:example-stripe-removal-in-fourier-space}}\begin{itemize}
\item {} 
\sphinxAtStartPar
Transform the image to Fourier space
\$\(\mathcal{F_{2D}}\left\{a\right\} \Rightarrow A\)\$

\item {} 
\sphinxAtStartPar
Multiply spectrum image by band pass filter
\$\(A_{filtered}=A \cdot H\)\$

\item {} 
\sphinxAtStartPar
Compute the inverse transform to obtain the filtered image in real space
\$\(\mathcal{{F_{2D}}^{-1}}\left\{A_{filtered}\right\} \Rightarrow a_{filtered}\)\$

\end{itemize}

\begin{sphinxVerbatim}[commandchars=\\\{\}]
\PYG{n}{plt}\PYG{o}{.}\PYG{n}{figure}\PYG{p}{(}\PYG{n}{figsize}\PYG{o}{=}\PYG{p}{[}\PYG{l+m+mi}{10}\PYG{p}{,}\PYG{l+m+mi}{8}\PYG{p}{]}\PYG{p}{)}
\PYG{n}{plt}\PYG{o}{.}\PYG{n}{subplot}\PYG{p}{(}\PYG{l+m+mi}{4}\PYG{p}{,}\PYG{l+m+mi}{1}\PYG{p}{,}\PYG{l+m+mi}{1}\PYG{p}{)}\PYG{p}{;}\PYG{n}{plt}\PYG{o}{.}\PYG{n}{imshow}\PYG{p}{(}\PYG{n}{plt}\PYG{o}{.}\PYG{n}{imread}\PYG{p}{(}\PYG{l+s+s1}{\PYGZsq{}}\PYG{l+s+s1}{figures/raw\PYGZus{}img.png}\PYG{l+s+s1}{\PYGZsq{}}\PYG{p}{)}\PYG{p}{)}\PYG{p}{;} \PYG{n}{plt}\PYG{o}{.}\PYG{n}{title}\PYG{p}{(}\PYG{l+s+s1}{\PYGZsq{}}\PYG{l+s+s1}{\PYGZdl{}a\PYGZdl{}}\PYG{l+s+s1}{\PYGZsq{}}\PYG{p}{)}\PYG{p}{;} \PYG{n}{plt}\PYG{o}{.}\PYG{n}{axis}\PYG{p}{(}\PYG{l+s+s1}{\PYGZsq{}}\PYG{l+s+s1}{off}\PYG{l+s+s1}{\PYGZsq{}}\PYG{p}{)}\PYG{p}{;}
\PYG{n}{plt}\PYG{o}{.}\PYG{n}{subplot}\PYG{p}{(}\PYG{l+m+mi}{4}\PYG{p}{,}\PYG{l+m+mi}{1}\PYG{p}{,}\PYG{l+m+mi}{2}\PYG{p}{)}\PYG{p}{;}\PYG{n}{plt}\PYG{o}{.}\PYG{n}{imshow}\PYG{p}{(}\PYG{n}{plt}\PYG{o}{.}\PYG{n}{imread}\PYG{p}{(}\PYG{l+s+s1}{\PYGZsq{}}\PYG{l+s+s1}{figures/raw\PYGZus{}spec.png}\PYG{l+s+s1}{\PYGZsq{}}\PYG{p}{)}\PYG{p}{)}\PYG{p}{;} \PYG{n}{plt}\PYG{o}{.}\PYG{n}{title}\PYG{p}{(}\PYG{l+s+s1}{\PYGZsq{}}\PYG{l+s+s1}{\PYGZdl{}}\PYG{l+s+s1}{\PYGZbs{}}\PYG{l+s+s1}{mathcal}\PYG{l+s+si}{\PYGZob{}F\PYGZcb{}}\PYG{l+s+s1}{(a)\PYGZdl{}}\PYG{l+s+s1}{\PYGZsq{}}\PYG{p}{)}\PYG{p}{;} \PYG{n}{plt}\PYG{o}{.}\PYG{n}{axis}\PYG{p}{(}\PYG{l+s+s1}{\PYGZsq{}}\PYG{l+s+s1}{off}\PYG{l+s+s1}{\PYGZsq{}}\PYG{p}{)}\PYG{p}{;}
\PYG{n}{plt}\PYG{o}{.}\PYG{n}{subplot}\PYG{p}{(}\PYG{l+m+mi}{4}\PYG{p}{,}\PYG{l+m+mi}{1}\PYG{p}{,}\PYG{l+m+mi}{3}\PYG{p}{)}\PYG{p}{;}\PYG{n}{plt}\PYG{o}{.}\PYG{n}{imshow}\PYG{p}{(}\PYG{n}{plt}\PYG{o}{.}\PYG{n}{imread}\PYG{p}{(}\PYG{l+s+s1}{\PYGZsq{}}\PYG{l+s+s1}{figures/filt\PYGZus{}spec.png}\PYG{l+s+s1}{\PYGZsq{}}\PYG{p}{)}\PYG{p}{)}\PYG{p}{;} \PYG{n}{plt}\PYG{o}{.}\PYG{n}{title}\PYG{p}{(}\PYG{l+s+s1}{\PYGZsq{}}\PYG{l+s+s1}{Kernel \PYGZdl{}H\PYGZdl{}}\PYG{l+s+s1}{\PYGZsq{}}\PYG{p}{)}\PYG{p}{;} \PYG{n}{plt}\PYG{o}{.}\PYG{n}{axis}\PYG{p}{(}\PYG{l+s+s1}{\PYGZsq{}}\PYG{l+s+s1}{off}\PYG{l+s+s1}{\PYGZsq{}}\PYG{p}{)}\PYG{p}{;}
\PYG{n}{plt}\PYG{o}{.}\PYG{n}{subplot}\PYG{p}{(}\PYG{l+m+mi}{4}\PYG{p}{,}\PYG{l+m+mi}{1}\PYG{p}{,}\PYG{l+m+mi}{4}\PYG{p}{)}\PYG{p}{;}\PYG{n}{plt}\PYG{o}{.}\PYG{n}{imshow}\PYG{p}{(}\PYG{n}{plt}\PYG{o}{.}\PYG{n}{imread}\PYG{p}{(}\PYG{l+s+s1}{\PYGZsq{}}\PYG{l+s+s1}{figures/filt\PYGZus{}img.png}\PYG{l+s+s1}{\PYGZsq{}}\PYG{p}{)}\PYG{p}{)}\PYG{p}{;} \PYG{n}{plt}\PYG{o}{.}\PYG{n}{title}\PYG{p}{(}\PYG{l+s+s1}{\PYGZsq{}}\PYG{l+s+s1}{\PYGZdl{}a\PYGZus{}}\PYG{l+s+si}{\PYGZob{}filtered\PYGZcb{}}\PYG{l+s+s1}{\PYGZdl{}}\PYG{l+s+s1}{\PYGZsq{}}\PYG{p}{)}\PYG{p}{;} \PYG{n}{plt}\PYG{o}{.}\PYG{n}{axis}\PYG{p}{(}\PYG{l+s+s1}{\PYGZsq{}}\PYG{l+s+s1}{off}\PYG{l+s+s1}{\PYGZsq{}}\PYG{p}{)}\PYG{p}{;}
\end{sphinxVerbatim}

\noindent\sphinxincludegraphics{{02-ImageEnhancement_137_0}.png}


\section{The effect of the stripe filter}
\label{\detokenize{02-ImageEnhancement:the-effect-of-the-stripe-filter}}
\begin{figure}[htbp]
\centering
\capstart

\noindent\sphinxincludegraphics[scale=0.5]{{slice_with_stripes}.png}
\caption{CT slice before stripe removal filter.}\label{\detokenize{02-ImageEnhancement:id15}}\end{figure}

\begin{figure}[htbp]
\centering
\capstart

\noindent\sphinxincludegraphics[scale=0.5]{{slice_stripe_filtered}.png}
\caption{CT slice after applying stripe removal filter.}\label{\detokenize{02-ImageEnhancement:id16}}\end{figure}



\sphinxAtStartPar
Intensity variations are suppressed using the stripe filter on all projections.


\section{Technical details on Fourier space filters}
\label{\detokenize{02-ImageEnhancement:technical-details-on-fourier-space-filters}}

\subsection{When should you use convolution in Fourier space?}
\label{\detokenize{02-ImageEnhancement:when-should-you-use-convolution-in-fourier-space}}\begin{itemize}
\item {} 
\sphinxAtStartPar
Simplicity

\item {} 
\sphinxAtStartPar
Kernel size

\item {} 
\sphinxAtStartPar
Speed at repeated convolutions

\end{itemize}


\subsection{Zero padding}
\label{\detokenize{02-ImageEnhancement:zero-padding}}
\sphinxAtStartPar
The FFT is only working with data of size in \(2^N\). If your data has a different length, you have to pad (fill with constant value) up the next \(2^N\).


\section{Python functions}
\label{\detokenize{02-ImageEnhancement:python-functions}}

\subsection{Filters in the spatial domain}
\label{\detokenize{02-ImageEnhancement:filters-in-the-spatial-domain}}
\sphinxAtStartPar
e.g. from \sphinxcode{\sphinxupquote{scipy import ndimage}}
\begin{itemize}
\item {} 
\sphinxAtStartPar
\sphinxcode{\sphinxupquote{ndimage.filters.convolve(f,h)}} Linear filter using kernel \(h\) on image \(f\).

\item {} 
\sphinxAtStartPar
\sphinxcode{\sphinxupquote{ndimage.filters.median\_filter(f,\textbackslash{}{[}n,m\textbackslash{}{]})}}  Median filter using an \(n \times m\) filter neighborhood

\end{itemize}


\subsection{Fourier transform}
\label{\detokenize{02-ImageEnhancement:fourier-transform}}\begin{itemize}
\item {} 
\sphinxAtStartPar
\sphinxcode{\sphinxupquote{np.fft.fft2(f)}} Computes the 2D Fast Fourier Transform of image \(f\)

\item {} 
\sphinxAtStartPar
\sphinxcode{\sphinxupquote{np.fft.ifft2(F)}} Computes the inverse Fast Fourier Transform \(F\).

\item {} 
\sphinxAtStartPar
\sphinxcode{\sphinxupquote{np.fft.fftshift()}} Rearranges the data to center the \(\omega\)=0. Works for 1D and 2D.

\end{itemize}


\subsection{Complex numbers}
\label{\detokenize{02-ImageEnhancement:complex-numbers}}\begin{itemize}
\item {} 
\sphinxAtStartPar
\sphinxcode{\sphinxupquote{np.abs(f), np.angle(f)}} Computes amplitude and argument of a complex number.

\item {} 
\sphinxAtStartPar
\sphinxcode{\sphinxupquote{np.real(f), np.imag(f)}} Gives the real and imaginary parts of a complex number.

\end{itemize}


\chapter{Scale spaces}
\label{\detokenize{02-ImageEnhancement:scale-spaces}}

\section{Why scale spaces?}
\label{\detokenize{02-ImageEnhancement:why-scale-spaces}}

\subsection{Motivation}
\label{\detokenize{02-ImageEnhancement:motivation}}
\sphinxAtStartPar
Basic filters have problems to handle low SNR and textured noise.

\sphinxAtStartPar
Something new is required…


\subsection{The solution}
\label{\detokenize{02-ImageEnhancement:the-solution}}
\sphinxAtStartPar
Filtering on different scales can take noise suppression one step further.

\begin{figure}[htbp]
\centering
\capstart

\noindent\sphinxincludegraphics[scale=0.75]{{burt_pyramid}.pdf}
\caption{A scale pyramid of an image can be useful for filtering and segmentation.}\label{\detokenize{02-ImageEnhancement:id17}}\end{figure}




\section{Wavelets \sphinxhyphen{} the basic idea}
\label{\detokenize{02-ImageEnhancement:wavelets-the-basic-idea}}\begin{itemize}
\item {} 
\sphinxAtStartPar
The wavelet transform produces scales by decomposing a signal into two signals at a coarser scale containing \textbackslash{}emph\{trend\} and \textbackslash{}emph\{details\}.\textbackslash{}

\item {} 
\sphinxAtStartPar
The next scale is computed using the trend of the previous transform

\end{itemize}
\begin{equation*}
\begin{split}WT\{s\}\rightarrow\{a_1,d_1\}, WT\{a_1\}\rightarrow\{a_2,d_2\}, \ldots, WT\{a_{N-1}\}\rightarrow\{a_N,d_N\} \end{split}
\end{equation*}\begin{itemize}
\item {} 
\sphinxAtStartPar
The inverse transform brings \(s\) back using \(\{a_N,d_1, \ldots,d_N\}\).

\item {} 
\sphinxAtStartPar
Many wavelet bases exists, the choice depends on the application.

\end{itemize}


\subsection{Applications of wavelets}
\label{\detokenize{02-ImageEnhancement:applications-of-wavelets}}\begin{itemize}
\item {} 
\sphinxAtStartPar
Noise reduction

\item {} 
\sphinxAtStartPar
Analysis

\item {} 
\sphinxAtStartPar
Segmentation

\item {} 
\sphinxAtStartPar
Compression

\end{itemize}

\sphinxAtStartPar
\sphinxhref{https://doi.org/10.1201/9781584887461}{Walker 2008} \sphinxhref{https://doi.org/10.1016/B978-0-12-374370-1.X0001-8}{Mallat 2009}


\section{Wavelet transform of a 1D signal}
\label{\detokenize{02-ImageEnhancement:wavelet-transform-of-a-1d-signal}}
\begin{figure}[htbp]
\centering
\capstart

\noindent\sphinxincludegraphics[scale=0.6]{{wavelet1d_noaxis}.pdf}
\caption{A noisy test signal decomposed using the \sphinxstyleemphasis{symlet\sphinxhyphen{}4} wavelet base function.}\label{\detokenize{02-ImageEnhancement:id18}}\end{figure}



\sphinxAtStartPar
Using \sphinxstylestrong{symlet\sphinxhyphen{}4}


\section{Wavelet transform of an image}
\label{\detokenize{02-ImageEnhancement:wavelet-transform-of-an-image}}
\begin{figure}[htbp]
\centering
\capstart

\noindent\sphinxincludegraphics[scale=0.8]{{wt2d_schematic}.pdf}
\caption{Transform workflow for a 2D wavelet decomposition.}\label{\detokenize{02-ImageEnhancement:id19}}\end{figure}




\section{Wavelet transform of an image \sphinxhyphen{} example}
\label{\detokenize{02-ImageEnhancement:wavelet-transform-of-an-image-example}}
\begin{figure}[htbp]
\centering
\capstart

\noindent\sphinxincludegraphics[scale=0.4]{{testpattern_1024}.png}
\caption{Transform workflow for a 2D wavelet decomposition.}\label{\detokenize{02-ImageEnhancement:id20}}\end{figure}

\begin{figure}[htbp]
\centering
\capstart

\noindent\sphinxincludegraphics[scale=0.4]{{wt_testimage_gray}.png}
\caption{Transform workflow for a 2D wavelet decomposition.}\label{\detokenize{02-ImageEnhancement:id21}}\end{figure}




\section{Using wavelets for noise reduction}
\label{\detokenize{02-ImageEnhancement:using-wavelets-for-noise-reduction}}
\sphinxAtStartPar
The noise is found in the detail part of the WT
\begin{itemize}
\item {} 
\sphinxAtStartPar
Make a WT of the signal to a level that corresponds to the scale of the unwanted information.

\item {} 
\sphinxAtStartPar
Threshold the detail part \(d_{\gamma}=|d|<\gamma\quad ? \quad 0 : d\).

\item {} 
\sphinxAtStartPar
Inverse WT back to normal scale \(\rightarrow\) image is filtered.

\end{itemize}

\begin{figure}[htbp]
\centering
\capstart

\noindent\sphinxincludegraphics[scale=1.0]{{wavelet1d_filter}.pdf}
\caption{The principle of a basic noise reduction filter using wavelets.}\label{\detokenize{02-ImageEnhancement:id22}}\end{figure}




\section{Wavelet noise reduction \sphinxhyphen{} Image example}
\label{\detokenize{02-ImageEnhancement:wavelet-noise-reduction-image-example}}
\sphinxAtStartPar
Example Filtered using two levels of the Symlet\sphinxhyphen{}2 wavelet

\begin{figure}[htbp]
\centering
\capstart

\noindent\sphinxincludegraphics[scale=1.0]{{wavelet2d_filtered}.pdf}
\caption{An example of noise reduction using a wavelet filter.}\label{\detokenize{02-ImageEnhancement:id23}}\end{figure}

\sphinxAtStartPar
\sphinxstylestrong{Data:} Neutron CT of a lead scroll




\section{Python functions for wavelets}
\label{\detokenize{02-ImageEnhancement:python-functions-for-wavelets}}\begin{itemize}
\item {} 
\sphinxAtStartPar
\sphinxstylestrong{dwt2}/\sphinxstylestrong{idtw2} Makes one level of the wavelet transform or its inverse using wavelet base specified by ‘wn’.

\item {} 
\sphinxAtStartPar
\sphinxstylestrong{wavedec2} Performs N levels of wavelet decomposition using a specified wavelet base.

\item {} 
\sphinxAtStartPar
\sphinxstylestrong{wbmpen} Estimating threshold parameters for wavelet denoising.

\item {} 
\sphinxAtStartPar
\sphinxstylestrong{wdencmp} Wavelet denoising and compression using information from \sphinxstyleemphasis{wavedec2} and *wbmpen.

\end{itemize}


\chapter{Parameterized scale spaces}
\label{\detokenize{02-ImageEnhancement:parameterized-scale-spaces}}

\section{PDE based scale space filters}
\label{\detokenize{02-ImageEnhancement:pde-based-scale-space-filters}}
\begin{figure}[htbp]
\centering
\capstart

\noindent\sphinxincludegraphics[scale=0.5]{{slice_original}.pdf}
\caption{Noisy slice to be filtered.}\label{\detokenize{02-ImageEnhancement:id24}}\end{figure}

\begin{figure}[htbp]
\centering
\capstart

\noindent\sphinxincludegraphics[scale=0.5]{{slice_diffusion}.pdf}
\caption{Slice after diffusion filter.}\label{\detokenize{02-ImageEnhancement:id25}}\end{figure}

\begin{figure}[htbp]
\centering
\capstart

\noindent\sphinxincludegraphics[scale=0.5]{{slice_iss}.pdf}
\caption{Slice after ISS filter.}\label{\detokenize{02-ImageEnhancement:id26}}\end{figure}



\sphinxAtStartPar
These filters may work for applications where Linear and Rank filters fail.

\sphinxAtStartPar
\sphinxhref{https://doi.org/10.1007/978-0-387-44588-5}{Aubert 2002}.


\section{The starting point}
\label{\detokenize{02-ImageEnhancement:the-starting-point}}
\sphinxAtStartPar
The heat transport equation
\$\(\frac{\partial T}{\partial t}=\kappa \nabla^2 T\)\$
\begin{itemize}
\item {} 
\sphinxAtStartPar
\sphinxstylestrong{\(T\)} Image to filter (intensity \(\equiv\) temperature)

\item {} 
\sphinxAtStartPar
\sphinxstylestrong{\(\kappa\)} Thermal conduction capacity

\end{itemize}






\section{Controlling the diffusivity}
\label{\detokenize{02-ImageEnhancement:controlling-the-diffusivity}}
\begin{sphinxVerbatim}[commandchars=\\\{\}]
\PYG{k}{def} \PYG{n+nf}{g}\PYG{p}{(}\PYG{n}{x}\PYG{p}{,}\PYG{n}{lambd}\PYG{p}{,}\PYG{n}{n}\PYG{p}{)} \PYG{p}{:}
    \PYG{n}{g}\PYG{o}{=}\PYG{l+m+mi}{1}\PYG{o}{/}\PYG{p}{(}\PYG{l+m+mi}{1}\PYG{o}{+}\PYG{p}{(}\PYG{n}{x}\PYG{o}{/}\PYG{n}{lambd}\PYG{p}{)}\PYG{o}{*}\PYG{o}{*}\PYG{n}{n}\PYG{p}{)}
    \PYG{k}{return} \PYG{n}{g}
\end{sphinxVerbatim}

\sphinxAtStartPar
We want to control the diffusion process…
\begin{itemize}
\item {} 
\sphinxAtStartPar
\sphinxstyleemphasis{Near edges} The Diffusivity \(\rightarrow\) 0

\item {} 
\sphinxAtStartPar
\sphinxstyleemphasis{Flat regions} The Diffusivity \(\rightarrow\) 1

\end{itemize}

\sphinxAtStartPar
The contrast function \(G\) is our control function
\$\(G(x)=\frac{1}{1+\left(\frac{x}{\lambda}\right)^n}\)\$
\begin{itemize}
\item {} 
\sphinxAtStartPar
\sphinxstyleemphasis{\(\lambda\)} Threshold level

\item {} 
\sphinxAtStartPar
\sphinxstyleemphasis{\(n\)} Steepness of the threshold function

\end{itemize}

\begin{sphinxVerbatim}[commandchars=\\\{\}]
\PYG{n}{x}\PYG{o}{=}\PYG{n}{np}\PYG{o}{.}\PYG{n}{linspace}\PYG{p}{(}\PYG{l+m+mi}{0}\PYG{p}{,}\PYG{l+m+mi}{1}\PYG{p}{,}\PYG{l+m+mi}{100}\PYG{p}{)}\PYG{p}{;}

\PYG{n}{plt}\PYG{o}{.}\PYG{n}{plot}\PYG{p}{(}\PYG{n}{x}\PYG{p}{,}\PYG{n}{g}\PYG{p}{(}\PYG{n}{x}\PYG{p}{,}\PYG{n}{lambd}\PYG{o}{=}\PYG{l+m+mf}{0.5}\PYG{p}{,}\PYG{n}{n}\PYG{o}{=}\PYG{l+m+mi}{12}\PYG{p}{)}\PYG{p}{)}\PYG{p}{;}
\PYG{n}{plt}\PYG{o}{.}\PYG{n}{xlabel}\PYG{p}{(}\PYG{l+s+s1}{\PYGZsq{}}\PYG{l+s+s1}{Image intensity (x)}\PYG{l+s+s1}{\PYGZsq{}}\PYG{p}{)}\PYG{p}{;} \PYG{n}{plt}\PYG{o}{.}\PYG{n}{ylabel}\PYG{p}{(}\PYG{l+s+s1}{\PYGZsq{}}\PYG{l+s+s1}{G(x)}\PYG{l+s+s1}{\PYGZsq{}}\PYG{p}{)}\PYG{p}{;}\PYG{n}{plt}\PYG{o}{.}\PYG{n}{tight\PYGZus{}layout}\PYG{p}{(}\PYG{p}{)}
\end{sphinxVerbatim}

\noindent\sphinxincludegraphics{{02-ImageEnhancement_173_0}.png}


\section{Gradient controlled diffusivity}
\label{\detokenize{02-ImageEnhancement:gradient-controlled-diffusivity}}\begin{equation*}
\begin{split}\frac{\partial u}{\partial t}=G(|\nabla u|)\nabla^2 u\end{split}
\end{equation*}
\begin{sphinxVerbatim}[commandchars=\\\{\}]
\PYG{n}{plt}\PYG{o}{.}\PYG{n}{subplot}\PYG{p}{(}\PYG{l+m+mi}{1}\PYG{p}{,}\PYG{l+m+mi}{2}\PYG{p}{,}\PYG{l+m+mi}{1}\PYG{p}{)}\PYG{p}{;} \PYG{n}{plt}\PYG{o}{.}\PYG{n}{imshow}\PYG{p}{(}\PYG{n}{io}\PYG{o}{.}\PYG{n}{imread}\PYG{p}{(}\PYG{l+s+s2}{\PYGZdq{}}\PYG{l+s+s2}{figures/aggregates.png}\PYG{l+s+s2}{\PYGZdq{}}\PYG{p}{)}\PYG{p}{,}\PYG{n}{cmap}\PYG{o}{=}\PYG{l+s+s1}{\PYGZsq{}}\PYG{l+s+s1}{gray}\PYG{l+s+s1}{\PYGZsq{}}\PYG{p}{)}\PYG{p}{;} \PYG{n}{plt}\PYG{o}{.}\PYG{n}{title}\PYG{p}{(}\PYG{l+s+s1}{\PYGZsq{}}\PYG{l+s+s1}{Image}\PYG{l+s+s1}{\PYGZsq{}}\PYG{p}{)}\PYG{p}{;}
\PYG{n}{plt}\PYG{o}{.}\PYG{n}{subplot}\PYG{p}{(}\PYG{l+m+mi}{1}\PYG{p}{,}\PYG{l+m+mi}{2}\PYG{p}{,}\PYG{l+m+mi}{2}\PYG{p}{)}\PYG{p}{;} \PYG{n}{plt}\PYG{o}{.}\PYG{n}{imshow}\PYG{p}{(}\PYG{n}{io}\PYG{o}{.}\PYG{n}{imread}\PYG{p}{(}\PYG{l+s+s2}{\PYGZdq{}}\PYG{l+s+s2}{figures/diffusivity.png}\PYG{l+s+s2}{\PYGZdq{}}\PYG{p}{)}\PYG{p}{,}\PYG{n}{cmap}\PYG{o}{=}\PYG{l+s+s1}{\PYGZsq{}}\PYG{l+s+s1}{gray}\PYG{l+s+s1}{\PYGZsq{}}\PYG{p}{)}\PYG{p}{;} \PYG{n}{plt}\PYG{o}{.}\PYG{n}{title}\PYG{p}{(}\PYG{l+s+s1}{\PYGZsq{}}\PYG{l+s+s1}{Diffusivity map}\PYG{l+s+s1}{\PYGZsq{}}\PYG{p}{)}\PYG{p}{;}
\end{sphinxVerbatim}

\noindent\sphinxincludegraphics{{02-ImageEnhancement_175_0}.png}
\begin{itemize}
\item {} 
\sphinxAtStartPar
\sphinxstyleemphasis{\(u\)} Image to be filtered

\item {} 
\sphinxAtStartPar
\sphinxstyleemphasis{\(G(\cdot)\)} Non\sphinxhyphen{}linear function to control the diffusivity

\item {} 
\sphinxAtStartPar
\sphinxstyleemphasis{\(\tau\)} Time increment

\item {} 
\sphinxAtStartPar
\sphinxstyleemphasis{\(N\)} Number of iterations

\end{itemize}




\section{The non\sphinxhyphen{}linear diffusion filter}
\label{\detokenize{02-ImageEnhancement:the-non-linear-diffusion-filter}}
\sphinxAtStartPar
A more robust filter is obtained with
\begin{equation*}
\begin{split}\frac{\partial u}{\partial t}=G(|\nabla_{\sigma} u|)\nabla^2u\end{split}
\end{equation*}\begin{itemize}
\item {} 
\sphinxAtStartPar
\sphinxstyleemphasis{\(u\)} Image to be filtered

\item {} 
\sphinxAtStartPar
\sphinxstyleemphasis{\(G(\cdot)\)} Non\sphinxhyphen{}linear function to control the contrast

\item {} 
\sphinxAtStartPar
\sphinxstyleemphasis{\(\tau\)} Time increment per numerical iteration

\item {} 
\sphinxAtStartPar
\sphinxstyleemphasis{\(N\)} Number of iterations

\item {} 
\sphinxAtStartPar
\sphinxstyleemphasis{\(\nabla_{\sigma}\)} Gradient smoothed by a Gaussian filter, width \(\sigma\)

\end{itemize}


\section{Diffusion filter example}
\label{\detokenize{02-ImageEnhancement:diffusion-filter-example}}
\sphinxAtStartPar
Neutron CT slice from a real\sphinxhyphen{}time experiment observing the coalescence of cold mixed bitumen.




\section{Filtering as a regularization problem}
\label{\detokenize{02-ImageEnhancement:filtering-as-a-regularization-problem}}

\subsection{The continued development}
\label{\detokenize{02-ImageEnhancement:the-continued-development}}\begin{itemize}
\item {} 
\sphinxAtStartPar
\sphinxstylestrong{90’s} During the late 90’s the diffusion filter was described in terms of a regularization problem.

\item {} 
\sphinxAtStartPar
\sphinxstylestrong{00’s} Work toward regularization of total variation minimization.

\end{itemize}


\subsubsection{TV\sphinxhyphen{}L1}
\label{\detokenize{02-ImageEnhancement:tv-l1}}\begin{equation*}
\begin{split}u=\underset{u\in BV(\Omega)}{\operatorname{argmin}}\left\{\underbrace{|u|_{BV}}_{noise}+ \underbrace{\mbox{$\frac{\lambda}{2}$}\|f-u\|_{1}}_{fidelity}\right\}\end{split}
\end{equation*}

\subsubsection{Rudin\sphinxhyphen{}Osher\sphinxhyphen{}Fatemi model (ROF)}
\label{\detokenize{02-ImageEnhancement:rudin-osher-fatemi-model-rof}}\begin{equation*}
\begin{split}u=\underset{u\in BV(\Omega)}{\operatorname{argmin}}
\left\{
\underbrace{|u|_{BV}}_{noise} 
+ \underbrace{\mbox{$\frac{\lambda}{2}$} \|f-u\|^2_{2}}_{fidelity}
\right\}\end{split}
\end{equation*}
\sphinxAtStartPar
with \(|u|_{BV}=\int_{\Omega}|\nabla u|^2\)


\section{The inverse scale space filter}
\label{\detokenize{02-ImageEnhancement:the-inverse-scale-space-filter}}

\subsection{The idea}
\label{\detokenize{02-ImageEnhancement:the-idea}}
\sphinxAtStartPar
We want smooth regions with sharp edges\textbackslash{}ldots
\begin{itemize}
\item {} 
\sphinxAtStartPar
Turn the processing order of scale space filter upside down

\item {} 
\sphinxAtStartPar
Start with an empty image

\item {} 
\sphinxAtStartPar
Add large structures successively until an image with relevant features appears

\end{itemize}


\subsection{The ISS filter \sphinxhyphen{} Some properties}
\label{\detokenize{02-ImageEnhancement:the-iss-filter-some-properties}}\begin{itemize}
\item {} 
\sphinxAtStartPar
is an edge preserving filter for noise reduction.

\item {} 
\sphinxAtStartPar
is defined by a partial differential equation.

\item {} 
\sphinxAtStartPar
has a well defined termination point.

\end{itemize}

\sphinxAtStartPar
\sphinxhref{https://dx.doi.org/10.4310/CMS.2006.v4.n1.a7}{Burger et al. 2006}


\section{The ROF filter equation}
\label{\detokenize{02-ImageEnhancement:the-rof-filter-equation}}
\sphinxAtStartPar
The image \(f\) is filtered by solving

\sphinxAtStartPar
\(\frac{\partial{}u}{\partial{}t}=\mathrm{div}\left(\frac{\nabla{}u}{|\nabla{}u|}\right)+\lambda{}(f-u+v)\)

\sphinxAtStartPar
\(\frac{\partial{}v}{\partial{}t}=\alpha{}(f-u)\)


\subsection{Variables:}
\label{\detokenize{02-ImageEnhancement:variables}}\begin{itemize}
\item {} 
\sphinxAtStartPar
\sphinxstyleemphasis{\(f\)} Input image

\item {} 
\sphinxAtStartPar
\sphinxstyleemphasis{\(u\)} Filtered image

\item {} 
\sphinxAtStartPar
\sphinxstyleemphasis{\(v\)} Regularization term (feedback of previous iteration)

\end{itemize}


\subsection{Filter parameters}
\label{\detokenize{02-ImageEnhancement:filter-parameters}}\begin{itemize}
\item {} 
\sphinxAtStartPar
\sphinxstyleemphasis{\(\lambda\)} Related to the scale of the features to suppress.

\item {} 
\sphinxAtStartPar
\sphinxstyleemphasis{\(\alpha\)} Quality refinement

\item {} 
\sphinxAtStartPar
\sphinxstyleemphasis{\(N\)} Number of iterations

\item {} 
\sphinxAtStartPar
\sphinxstyleemphasis{\(\tau\)} Time increment

\end{itemize}


\section{Filter iterations}
\label{\detokenize{02-ImageEnhancement:filter-iterations}}
\sphinxAtStartPar
Neutron CT of dried lung filtered using 3D ISS filter




\section{How to choose \protect\(\lambda\protect\) and \protect\(\alpha\protect\)}
\label{\detokenize{02-ImageEnhancement:how-to-choose-lambda-and-alpha}}
\sphinxAtStartPar
The requirements varies between different data sets.


\subsection{Initial conditions:}
\label{\detokenize{02-ImageEnhancement:initial-conditions}}\begin{itemize}
\item {} 
\sphinxAtStartPar
Signal to noise ratio

\item {} 
\sphinxAtStartPar
Image features (fine grained or wide spread)

\end{itemize}


\subsection{Experiment:}
\label{\detokenize{02-ImageEnhancement:experiment}}\begin{itemize}
\item {} 
\sphinxAtStartPar
Scan \(\lambda\) and \(\alpha\)

\item {} 
\sphinxAtStartPar
Stop at \(T=n\, \tau= \sigma\) use different \(\tau\)

\item {} 
\sphinxAtStartPar
When does different effects occur, related to \(\sigma\)?

\end{itemize}


\section{Solutions at different times}
\label{\detokenize{02-ImageEnhancement:solutions-at-different-times}}
\sphinxAtStartPar
nini


\begin{savenotes}\sphinxattablestart
\centering
\begin{tabulary}{\linewidth}[t]{|T|T|T|T|T|T|T|}
\hline
\sphinxstyletheadfamily 
\sphinxAtStartPar
1
&\sphinxstyletheadfamily 
\sphinxAtStartPar
30
&\sphinxstyletheadfamily 
\sphinxAtStartPar
60
&\sphinxstyletheadfamily 
\sphinxAtStartPar
100
&\sphinxstyletheadfamily 
\sphinxAtStartPar
200
&\sphinxstyletheadfamily 
\sphinxAtStartPar
500
&\sphinxstyletheadfamily 
\sphinxAtStartPar
999
\\
\hline
\sphinxAtStartPar
\sphinxincludegraphics{{iss-iteration_0001}.png}
&
\sphinxAtStartPar
\sphinxincludegraphics{{iss-iteration_0030}.png}
&
\sphinxAtStartPar
\sphinxincludegraphics{{iss-iteration_0060}.png}
&
\sphinxAtStartPar
\sphinxincludegraphics{{iss-iteration_0100}.png}
&
\sphinxAtStartPar
\sphinxincludegraphics{{iss-iteration_0200}.png}
&
\sphinxAtStartPar
\sphinxincludegraphics{{iss-iteration_0500}.png}
&
\sphinxAtStartPar
\sphinxincludegraphics{{iss-iteration_0999}.png}
\\
\hline
\end{tabulary}
\par
\sphinxattableend\end{savenotes}



\begin{figure}[htbp]
\centering
\capstart

\noindent\sphinxincludegraphics[scale=0.7]{{iss-errorplot}.pdf}
\caption{Error plot for different solution times of the ISS filter.}\label{\detokenize{02-ImageEnhancement:id27}}\end{figure}


\section{Solution time}
\label{\detokenize{02-ImageEnhancement:solution-time}}
\sphinxAtStartPar
The solution time (\(T=N\cdot\tau\)) is essential to the result
\begin{itemize}
\item {} 
\sphinxAtStartPar
\sphinxstyleemphasis{\(\tau\) large} The solution is reached fast

\item {} 
\sphinxAtStartPar
\sphinxstyleemphasis{\(\tau\) small} The numerical accuracy is better

\end{itemize}


\section{The choice of initial image}
\label{\detokenize{02-ImageEnhancement:the-choice-of-initial-image}}
\begin{figure}[htbp]
\centering
\capstart

\noindent\sphinxincludegraphics[scale=0.7]{{iss-errorplots_initialimage}.pdf}
\caption{Error plots depend on the choice of the initial image.}\label{\detokenize{02-ImageEnhancement:id28}}\end{figure}



\sphinxAtStartPar
At some \(T\) the solution with \(u_0=f\) and \(u_0=0\) converge.


\chapter{Non\sphinxhyphen{}local means}
\label{\detokenize{02-ImageEnhancement:non-local-means}}

\section{Non\sphinxhyphen{}local smoothing}
\label{\detokenize{02-ImageEnhancement:non-local-smoothing}}

\subsection{The idea}
\label{\detokenize{02-ImageEnhancement:id2}}
\sphinxAtStartPar
Smoothing normally consider information from the neighborhood like
\begin{itemize}
\item {} 
\sphinxAtStartPar
Local averages (convolution)

\item {} 
\sphinxAtStartPar
Gradients and Curvatures (PDE filters)

\end{itemize}

\sphinxAtStartPar
Non\sphinxhyphen{}local smoothing average similiar intensities in a global sense.
\begin{itemize}
\item {} 
\sphinxAtStartPar
Every filtered pixel is a weighted average of all pixels.

\item {} 
\sphinxAtStartPar
Weights computed using difference between pixel intensities.

\end{itemize}

\sphinxAtStartPar
\sphinxhref{http://dx.doi.org/10.1137/040616024}{Buades et al. 2005}


\section{Filter definition}
\label{\detokenize{02-ImageEnhancement:filter-definition}}
\sphinxAtStartPar
The non\sphinxhyphen{}local means filter is defined as
\$\(u(p)=\frac{1}{C(p)}\sum_{q\in\Omega}v(q)\,f(p,q)\)\$
where
\begin{itemize}
\item {} 
\sphinxAtStartPar
\sphinxstyleemphasis{\(v\) and \(u\)} input and result images.

\item {} 
\sphinxAtStartPar
\sphinxstyleemphasis{\(C(p)\)} is the sum of all pixel weights as

\sphinxAtStartPar
\(C(p)=\sum_{q\in\Omega}f(p,q)\)

\item {} 
\sphinxAtStartPar
\sphinxstyleemphasis{\(f(p,q)\)} is the weighting function

\sphinxAtStartPar
\(f(p,q)=e^{-\frac{|B(q)-B(p)|^2}{h^2}}\)

\item {} 
\sphinxAtStartPar
\sphinxstyleemphasis{B(x)} is a neighborhood operator e.g. local average around \(x\)

\end{itemize}


\section{Non\sphinxhyphen{}local means 2D \sphinxhyphen{} Example}
\label{\detokenize{02-ImageEnhancement:non-local-means-2d-example}}
\begin{figure}[htbp]
\centering
\capstart

\noindent\sphinxincludegraphics[scale=1.0]{{nonlocal_example}.png}
\caption{Demonstration on the non\sphinxhyphen{}local means filter.}\label{\detokenize{02-ImageEnhancement:id29}}\end{figure}




\subsection{Observations}
\label{\detokenize{02-ImageEnhancement:observations}}\begin{itemize}
\item {} 
\sphinxAtStartPar
Good smoothing effect.

\item {} 
\sphinxAtStartPar
Strong thin lines are preserved.

\item {} 
\sphinxAtStartPar
Some patchiness related to filter parameter \(t\), \sphinxstyleemphasis{i.e.} the size of \(\Omega_i\).

\end{itemize}


\section{Performance complications}
\label{\detokenize{02-ImageEnhancement:performance-complications}}

\subsection{Problem}
\label{\detokenize{02-ImageEnhancement:id3}}
\sphinxAtStartPar
The orignal filter compares all pixels with all pixels\textbackslash{}ldots\textbackslash{}
\begin{itemize}
\item {} 
\sphinxAtStartPar
Complexity \(\mathcal{O}(N^2)\)

\item {} 
\sphinxAtStartPar
Not feasible for large images, and particular 3D images!

\end{itemize}


\subsection{Solution}
\label{\detokenize{02-ImageEnhancement:solution}}
\sphinxAtStartPar
It has been shown that not all pixels have to be compared to achieve a good filter effect.
i.e. \(\Omega\) in the filter equations can be replaced by \(\Omega_i<<\Omega\)


\chapter{Verification}
\label{\detokenize{02-ImageEnhancement:verification}}

\section{How good is my filter?}
\label{\detokenize{02-ImageEnhancement:how-good-is-my-filter}}

\section{Verify the correctness of the method}
\label{\detokenize{02-ImageEnhancement:verify-the-correctness-of-the-method}}

\subsection{“Data massage”}
\label{\detokenize{02-ImageEnhancement:data-massage}}
\sphinxAtStartPar
Filtering manipulates the data, avoid too strong modifications otherwise you may invent new image features!!!

\begin{figure}[htbp]
\centering
\capstart

\noindent\sphinxincludegraphics[scale=0.6]{{mugs}.png}
\caption{Be careful when you apply different filters. You may make to great modifications.}\label{\detokenize{02-ImageEnhancement:id30}}\end{figure}




\subsection{Verify the validity your method}
\label{\detokenize{02-ImageEnhancement:verify-the-validity-your-method}}\begin{itemize}
\item {} 
\sphinxAtStartPar
Visual inspection

\item {} 
\sphinxAtStartPar
Difference images

\item {} 
\sphinxAtStartPar
Use degraded phantom images in a “smoke test”

\end{itemize}


\section{Verification using difference images}
\label{\detokenize{02-ImageEnhancement:verification-using-difference-images}}
\sphinxAtStartPar
Compute pixel\sphinxhyphen{}wise difference between image \(f\) and \(g\)


\sphinxAtStartPar
Difference images provide first diagnosis about processing performance


\section{Performance testing \sphinxhyphen{} The smoke test}
\label{\detokenize{02-ImageEnhancement:performance-testing-the-smoke-test}}\begin{itemize}
\item {} 
\sphinxAtStartPar
Testing term from electronic hardware testing \sphinxhyphen{} drive the system until something fails due to overheating…

\item {} 
\sphinxAtStartPar
In general: scan the parameter space for different SNR until the method fails to identify strength and weakness of the system.

\end{itemize}


\subsection{Test strategy}
\label{\detokenize{02-ImageEnhancement:test-strategy}}\begin{enumerate}
\sphinxsetlistlabels{\arabic}{enumi}{enumii}{}{.}%
\item {} 
\sphinxAtStartPar
Create a phantom image with relevant features.

\item {} 
\sphinxAtStartPar
Add noise for different SNR to the phantom.

\item {} 
\sphinxAtStartPar
Apply the processing method with different parameters.

\item {} 
\sphinxAtStartPar
Measure the difference between processed and phantom.

\item {} 
\sphinxAtStartPar
Repeat steps 2\sphinxhyphen{}4 \(N\) times for better test statistics.

\item {} 
\sphinxAtStartPar
Plot the results and identify the range of SNR and parameters that produce acceptable results.

\end{enumerate}


\section{Data for evaluation \sphinxhyphen{}  Phantom data}
\label{\detokenize{02-ImageEnhancement:data-for-evaluation-phantom-data}}
\sphinxAtStartPar
General purpose can be controlled
\begin{itemize}
\item {} 
\sphinxAtStartPar
Data with known features.

\item {} 
\sphinxAtStartPar
Parameters can be changed.
\begin{itemize}
\item {} 
\sphinxAtStartPar
Shape

\item {} 
\sphinxAtStartPar
Sharpness

\item {} 
\sphinxAtStartPar
Contrast

\item {} 
\sphinxAtStartPar
Noise (distribution and strength)

\end{itemize}

\end{itemize}



\begin{figure}[htbp]
\centering
\capstart

\noindent\sphinxincludegraphics[scale=0.5]{{shepplogan}.png}
\caption{The  shepp logan phantom.}\label{\detokenize{02-ImageEnhancement:id31}}\end{figure}

\begin{figure}[htbp]
\centering
\capstart

\noindent\sphinxincludegraphics[scale=0.5]{{synthetic_root4}.png}
\caption{A simulated root network.}\label{\detokenize{02-ImageEnhancement:id32}}\end{figure}


\section{Data for evaluation \sphinxhyphen{} Labelled data}
\label{\detokenize{02-ImageEnhancement:data-for-evaluation-labelled-data}}
\sphinxAtStartPar
Often ‘real’ data
\begin{itemize}
\item {} 
\sphinxAtStartPar
Labeled by experts

\item {} 
\sphinxAtStartPar
Used for training and validation
\begin{itemize}
\item {} 
\sphinxAtStartPar
Training of model

\item {} 
\sphinxAtStartPar
Validation

\item {} 
\sphinxAtStartPar
Test

\end{itemize}

\end{itemize}



\begin{figure}[htbp]
\centering
\capstart

\noindent\sphinxincludegraphics[scale=0.75]{{mnist}.png}
\caption{Images of hand written numbers from the MNIST data base.}\label{\detokenize{02-ImageEnhancement:id33}}\end{figure}


\section{Evaluation metrics for images}
\label{\detokenize{02-ImageEnhancement:evaluation-metrics-for-images}}
\sphinxAtStartPar
An evaluation procedure need a metric to compare the performance


\subsection{Mean squared error}
\label{\detokenize{02-ImageEnhancement:mean-squared-error}}



\subsection{Structural similarity index}
\label{\detokenize{02-ImageEnhancement:structural-similarity-index}}

\begin{itemize}
\item {} 
\sphinxAtStartPar
\sphinxstyleemphasis{\(\mu_f\), \(\mu_g\)} Local mean of \(f\) and \(g\).

\item {} 
\sphinxAtStartPar
\sphinxstyleemphasis{\(\sigma_{fg}\)} Local correlation between \(f\) and \(g\).

\item {} 
\sphinxAtStartPar
\sphinxstyleemphasis{\(\sigma_f\), \(\sigma_g\)} Local standard deviation of \(f\) and \(g\).

\item {} 
\sphinxAtStartPar
\sphinxstyleemphasis{\(C_1\), \(C_2\)} Constants based on the image dynamics (small numbers).

\end{itemize}



\sphinxAtStartPar
\sphinxhref{https://doi.org/10.1109/MSP.2008.930649}{Wang 2009}


\section{Test run example}
\label{\detokenize{02-ImageEnhancement:test-run-example}}
\sphinxAtStartPar
Runnig tests with different structure sizes and SNR.


\subsection{Phantom structure sizes}
\label{\detokenize{02-ImageEnhancement:phantom-structure-sizes}}

\begin{savenotes}\sphinxattablestart
\centering
\begin{tabulary}{\linewidth}[t]{|T|T|T|T|}
\hline
\sphinxstyletheadfamily 
\sphinxAtStartPar
1
&\sphinxstyletheadfamily 
\sphinxAtStartPar
2
&\sphinxstyletheadfamily 
\sphinxAtStartPar
4
&\sphinxstyletheadfamily 
\sphinxAtStartPar
8
\\
\hline
\sphinxAtStartPar
\sphinxincludegraphics{{synthetic_root1}.png}
&
\sphinxAtStartPar
\sphinxincludegraphics{{synthetic_root2}.png}
&
\sphinxAtStartPar
\sphinxincludegraphics{{synthetic_root4}.png}
&
\sphinxAtStartPar
\sphinxincludegraphics{{synthetic_root8}.png}
\\
\hline
\end{tabulary}
\par
\sphinxattableend\end{savenotes}


\subsection{Change SNR and contrast}
\label{\detokenize{02-ImageEnhancement:change-snr-and-contrast}}

\begin{savenotes}\sphinxattablestart
\centering
\begin{tabulary}{\linewidth}[t]{|T|T|T|T|}
\hline
\sphinxstyletheadfamily 
\sphinxAtStartPar
\(\sigma\)=1
&\sphinxstyletheadfamily 
\sphinxAtStartPar
\(\sigma\)=2
&\sphinxstyletheadfamily 
\sphinxAtStartPar
\(\sigma\)=5
&\sphinxstyletheadfamily 
\sphinxAtStartPar
\(\sigma\)=10
\\
\hline
\sphinxAtStartPar
\sphinxincludegraphics{{noise01}.png}
&
\sphinxAtStartPar
\sphinxincludegraphics{{noise02}.png}
&
\sphinxAtStartPar
\sphinxincludegraphics{{noise05}.png}
&
\sphinxAtStartPar
\sphinxincludegraphics{{noise10}.png}
\\
\hline
\end{tabulary}
\par
\sphinxattableend\end{savenotes}


\subsubsection{Process}
\label{\detokenize{02-ImageEnhancement:process}}



\subsubsection{Plot results}
\label{\detokenize{02-ImageEnhancement:plot-results}}

\begin{savenotes}\sphinxattablestart
\centering
\begin{tabulary}{\linewidth}[t]{|T|T|T|T|}
\hline
\sphinxstyletheadfamily 
\sphinxAtStartPar
R=1
&\sphinxstyletheadfamily 
\sphinxAtStartPar
R=2
&\sphinxstyletheadfamily 
\sphinxAtStartPar
R=4
&\sphinxstyletheadfamily 
\sphinxAtStartPar
R=8
\\
\hline
\sphinxAtStartPar
\sphinxincludegraphics{{radius1}.png}
&
\sphinxAtStartPar
\sphinxincludegraphics{{radius2}.png}
&
\sphinxAtStartPar
\sphinxincludegraphics{{radius4}.png}
&
\sphinxAtStartPar
\sphinxincludegraphics{{radius8}.png}
\\
\hline
\end{tabulary}
\par
\sphinxattableend\end{savenotes}

\sphinxAtStartPar
\sphinxhref{https://doi.org/10.1016/j.geoderma.2006.04.009}{Kaestner et al. 2006}


\chapter{Overview}
\label{\detokenize{02-ImageEnhancement:overview}}

\section{Many filters}
\label{\detokenize{02-ImageEnhancement:many-filters}}
\begin{figure}[htbp]
\centering
\capstart

\noindent\sphinxincludegraphics[scale=1.0]{{filter_overview_gray}.pdf}
\caption{All filters form the lecture for different SNR.}\label{\detokenize{02-ImageEnhancement:id34}}\end{figure}




\section{Details of filter performance}
\label{\detokenize{02-ImageEnhancement:details-of-filter-performance}}
\begin{figure}[htbp]
\centering
\capstart

\noindent\sphinxincludegraphics[scale=1.0]{{filter_overview_close_gray}.pdf}
\caption{A close\sphinxhyphen{}up of all filters form the lecture for different SNR.}\label{\detokenize{02-ImageEnhancement:id35}}\end{figure}




\section{Take\sphinxhyphen{}home message}
\label{\detokenize{02-ImageEnhancement:take-home-message}}
\sphinxAtStartPar
We have looked at different ways to suppress noise and artifacts:
\begin{itemize}
\item {} 
\sphinxAtStartPar
Convolution

\item {} 
\sphinxAtStartPar
Median filters

\item {} 
\sphinxAtStartPar
Wavelet denoising

\item {} 
\sphinxAtStartPar
PDE filters

\end{itemize}

\sphinxAtStartPar
Which one you select depends on
\begin{itemize}
\item {} 
\sphinxAtStartPar
Purpose of the data

\item {} 
\sphinxAtStartPar
Quality requirements

\item {} 
\sphinxAtStartPar
Available time

\end{itemize}









\renewcommand{\indexname}{Index}
\printindex
\end{document}